% -*-latex-*-
% Title: design.tex
% 
% Authors: Steve Parker and James Purciful
% 
% Synopsis: High level design for mapeditor 
%
% Last modified: 2-18-94
%%%%%%%%%%%%%%%%%%%%%%%%%

\documentstyle[12pt,epsf]{book}
%\newcommand{\X}[1]{{#1}\index{{#1}}}
\topmargin -0.45in
\textheight 9.0in
\oddsidemargin -.0in
\evensidemargin -.0in
\textwidth 6.5in

% Change the footnotes from numbers to a series of symbols
%\renewcommand{\thefootnote}{\fnsymbol{footnote}}

\begin{document}

% This is the ps include command:
% \hbox to \columnwidth{\hfil \epsffile{hw2a.eps} \hfil}

% Uncomment this line to get 1.5 spacing, replce 1.2 by 1.5 to get double
% spacing .  1.2 get 1 1/2 spacing
\setlength{\baselineskip}{1.2\baselineskip}

% Add a bit of space between each paragraph to make reading easier.
\setlength{\parskip}{\smallskipamount}


\title {High Level Design Document for\\
mapeditor}

\author {Steve Parker and James Purciful\thanks{Copyright \copyright 1994,
Steve Parker and James Purciful.  All rights reserved.}\\ 
Department of Computer Science\\ 
University of Utah\\ 
Salt Lake City, UT\\ 
email: \{sparker,purciful\}@cs.utah.edu}

\date {Winter Quarter 1994}



\maketitle

\tableofcontents



\chapter{Introduction}


\section{Overview of mapeditor}


\section{Purpose of this Document}


\section{Other Documentation}




\chapter{Internal Design Structure}


\section{Top Level}

\begin{figure}
\epsfysize=8cm
\hbox to \columnwidth{\hfil \epsffile{design.figures/toplevel.eps} \hfil}
\caption {Top Level Design of mapeditor}
\end{figure}

\chapter{Graphical User Interface}

\begin{figure}
\epsfysize=8cm
\hbox to \columnwidth{\hfil \epsffile{design.figures/icon3.ps} \hfil}
\caption {Graphical User Interface}
\end{figure}

\subsection{Responsibilities}

\subsection{Interface}

\subsection{Internal Structure}


\section{Map}

\subsection{Responsibilities}

\subsection{Interface}

\subsection{Internal Structure}
Mutex lock;
Array1<Module*> modules;
Array1<Wire*> wires;

\subsection{File Format}


\section{mapeditor}

\subsection{Responsibilities}

\subsection{Interface}

\subsection{Internal Structure}


\section{Scheduler}

The scheduler is a separate thread of control.  It is in charge of
deciding which module or modules are to be executing at any given
time.

\subsection{Responsibilities}

\subsection{Interface}

\subsection{Internal Structure}

\subsection{Scheduling Algorithm}

Initially, the scheduler will activate all modules simultaneously.  As
we develop experience with the behavior of modules, we can modify the
scheduler to take into account other factors, such as the number and
placement of processors, communications protocols, etc.

The scheduler is also able to start modules on a remote machine.  This
is performed by 

\section{Modules}

Each module is it's own thread of control.

\subsection{Responsibilities}

\subsection{Interface}

\subsection{Internal Structure}


\section{Miscellaneous}




\chapter{Inter-module Communication}


\section{Introduction}


\section{Inter-thread}


\section{Shared Memory}


\section{Distributed Shared Memory}


\section{Sockets}

\subsection{XDR}




\chapter{O/S Interface}


\section{Introduction}


\section{Solaris}


\section{Irix}



\chapter{Source code organization}

\section{Directories}

\section{Libraries}

\section{Executables}


%\newpage

\bibliographystyle{unsrt}
\bibliography{/home/sci/u/crj/bibtex_files/biglit,/home/sci/u/crj/bibtex_files/crj}


\end{document}




