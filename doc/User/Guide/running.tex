% -*-latex-*-
%
%  The contents of this file are subject to the University of Utah Public
%  License (the "License"); you may not use this file except in compliance
%  with the License.
%
%  Software distributed under the License is distributed on an "AS IS"
%  basis, WITHOUT WARRANTY OF ANY KIND, either express or implied. See the
%  License for the specific language governing rights and limitations under
%  the License.
%
%  The Original Source Code is SCIRun, released March 12, 2001.
%
%  The Original Source Code was developed by the University of Utah.
%  Portions created by UNIVERSITY are Copyright (C) 2001, 1994
%  University of Utah. All Rights Reserved.
%

% running.tex
%
% This is the `Running SCIRun' main section.

\section{Starting \sr{}}
\label{sec:startingup}

\subsection{The \command{scirun} Command}
\label{sec:sciruncmd}

%begin{latexonly}
\newcommand{\srwindow}%
{\centerline{\epsfig{file=Figures/srwindow-1.eps.gz, width=6in,
      bbllx=0, bblly=0, bburx=400, bbury=410}}}
%end{latexonly}
\begin{htmlonly}
  \newcommand{\srwindow}{%
    \htmladdimg[align=top,alt="SCIRun Window"]
    {../Figures/srwindow-1.gif}}
\end{htmlonly}


Start \sr{} by typing \keyboard{scirun} in a terminal (\eg \command{xterm})
window.  \note{Don't start \sr{} in the background, \ie don't type
  \keyboard{scirun \&}}.

The \command{scirun} command is located in \sr{}'s build directory
(see the
\htmladdnormallinkfoot{\srig}{\htmlurl{\latexhtml{\scisoftware/doc}{../../..}/Installation/Guide/ch.inst.html}}
for details).  The person who installed \sr{} can locate this command
for you.


\begin{figure}[htb]
  \begin{makeimage}
  \end{makeimage}
  \srwindow
  \caption{\label{fig:srwindow} \sr{} Main Window}
\end{figure}


Typing \keyboard{scirun} with no arguments starts up \sr{} with a blank \sr{}
window as shown in Figure~\ref{fig:srwindow}.  The main features of this
window are discussed in \secref{Anatomy of the Main
  Window}{sec:windowanatomy}.

The \command{scirun} command may take 1 argument
which is the name of a \sr{} \dfn{network} \index{network} file (these
files use a \filename{.net} extension).  These files hold previously
defined \sr{} networks.  \sr{} will load the specified network.  Network
files are discussed in a later section.

\sr{} may encounter errors during start up.  These will be displayed in
\sr{}'s error frame (see Figure~\ref{fig:srwindow}).  These errors
should be \htmladdnormallink{reported}{\bugsurl} to the \sr{} development
team.  \latexonly{See \secref{Reporting Bugs}{sec:bugs} for information on
  reporting bugs.}

\subsection{Anatomy of the Main Window}
\label{sec:windowanatomy}

The \sr{} main window consists of 4 main components (see
Figure~\ref{fig:srwindow}): 

\begin{description}
\item[Menu Bar] The menu bar is used to load networks, save networks, quit
  \sr{}, create network modules, and perform other tasks.  The menu bar
  consists of the following menu items:

  \begin{description}
  \item[\menu{File}] The \menu{File} menu contains the following items:
    \begin{description}
    \item[Save] Saves the network to a file (See \hyperref{this
        section}{Section~}{}{sec:savenet}). 
    \item[Save As...] Saves a network to a new file (See
      \hyperref{this section}{Section~}{}{sec:savenet}). 
    \item[Load] Loads a network from a file (See \hyperref{this
        section}{Section~}{}{sec:opennet}). 
    \item[Insert] Adds a network to the NetEdit frame without overlap
      (See \hyperref{this section}{Section~}{}{sec:insertnetwork}). 
    \item[Clear] Removes all modules and connections from the NetEdit
      frame (See \hyperref{this section}{Section~}{}{sec:clearnet}).
    \item[New] This sub-menu contains items of interest to developers only.
    \item[Add Info] Use this item to add network specific notes to
      the current network.  Notes should be used to document the purpose of
      the network (See \hyperref{this section}{Section~}{}{sec:docnetwork}).
    \item[Quit] Quits \sr{}.
    \end{description}
  \end{description}
  
  \begin{description}
  \item[\menu{SCIRun}] The \menu{SCIRun} menu is used to create modules
    (from the \sr{} package) for use in the net edit frame.  This menu is
    composed of sub-menus. Each sub-menu corresponds to a \dfn{category}
    \index{category} within the \sr{} package.  A category is a group of
    related modules.  Each menu item in a category sub-menu creates a
    specific module and places it in the netedit frame.  The netedit frame
    pop-up menu (activated by pressing the right mouse button when the
    mouse pointer is in the netedit frame) also provides access to the
    \menu{\sr{}} and \menu{\biopse{}} (and possibly other) package menus.  An
    overview of the contents of the \sr{} package is given in \secref{The
      \biopse{} Package}{sec:biopsepackage}.
  \end{description}

  \begin{description}
  \item[\menu{BioPSE}] The \menu{BioPSE} menu is used to create modules
    (from the \biopse package) for use in the netedit frame.  It consists
    of category sub-menus and module menu items.   An overview of the
    contents of the \biopse{} package is given in \secref{The SCIRun
      Package}{sec:srpackage}.
  \end{description}

  \begin{description}
  \item [\textit{Other Package Menus}] There may be other package
    menus if other packages have been installed.  They too will consist
    of category sub-menus and module menu items.
  \end{description}
  
\item[Global View Frame] The Global View Frame is located in the upper left
  corner of the main window (see Figure~\ref{fig:srwindow}). It is used to
  navigate complex networks.  The use of the Globla View Frame is
  described in \secref{Navigating a Network}{sec:navnetwork}.
  
\item[Error Frame] Errors during program startup are displayed in the Error
  Frame.  It is located in the upper right corner of the main window(see
  Figure~\ref{fig:srwindow}).  Errors on startup may mean that \sr{} has
  been installed incorrectly or has been installed from a buggy
  distribution.  Please report \hyperref{report}{(see
  Section~}{)}{sec:bugs} these 
  errors.
  
\item[NetEdit Frame] The NetEdit Frame occupies the bottom of the main
  window(see Figure~\ref{fig:srwindow}).  It is used to build and execute
  networks.  \secref{Building Networks}{sec:workwithnets} discusses the use
  of this frame.

\end{description}

\subsection{The Terminal Window}
\label{sec:termwinapp}

After starting, \sr{} will also run a shell-like application in the
terminal window called the \dfn{\sr{} shell}.  The \sr{} shell displays the
prompt \screen{scirun\ra}.  This program is actually a modified \dfna{Tool
  Command Language}{TCL} shell program and it is possible to type in
\acronym{TCL}'ish \sr{} commands at the prompt. The use of this shell
is described in \hyperref{a later section}{Section~}{}{sec:termapp}.


\subsection{Environment Variables}
\label{sec:environ} 
\index{Environment Variables}

\newcommand{\envitem}[1]{\item[\envvar{#1}]\mbox{}}

The following environment variables either affect \sr{}'s behavior or
improve its performance via remote connections.  With the exception of
\envvar{SCIRUN\_ON\_THE\_FLY\_LIBS\_DIR} the use of these variables is
optional.

\begin{description}
\envitem{SCIRUN\_ON\_THE\_FLY\_LIBS\_DIR}
  
  \envvar{SCIRUN\_ON\_THE\_FLY\_LIBS\_DIR} specifies the location of
  dynamically generated code.  See \secref{Dynamic
    Compilation}{sec:dyncomp} for details on the meaning and use of
  this variable.

\item[\envvar{SCIRUN\_DATA} \& \envvar{SCIRUN\_DATASET}]\mbox{}

  These two variables specify the complete path to a \sr{} \dfn{data
  set}.  A data set is a collection of \sr{} type data like fields,
  matricies, \etc{} that are stored in a directory and are used in
  \sr{} networks.

  \envvar{SCIRUN\_DATA} specifies a directory which contains a
  collection of data set directories.  \envvar{SCIRUN\_DATA} is used in
  conjunction with \envvar{SCIRUN\_DATASET} to specify the full path of
  a data set.  \envvar{SCIRUN\_DATASET} specifies a specific data set
  directory. 

  When working with the sample \sr{} data sets, for example,
  \envvar{SCIRUN\_DATA} would be set to
  \filename{/usr/local/SCIRunData} (assuming this is where \sr{}'s
  sample data were installed).  \envvar{SCIRUN\_DATASET} could then be
  set to \filename{utahtorso} to specify the Utah Torso data set.
  
\envitem{DISPLAY}
  
  If \sr{} is executed remotely then the value of the
  \envvar{DISPLAY} variable (as set on the remote machine) must
  be set correctly and the remote machine must be allowed to
  talk to the local X11 server.
  
  For telnet-like (unencrypted) connections to a remote machine
  you may set \envvar{DISPLAY} as follows:

\begin{verbatim}
  export DISPLAY=local-ip-addres:0.0
\end{verbatim}
  
  for an sh-style shell. And like this:

\begin{verbatim}
  setenv DISPLAY local-ip-addres:0.0
\end{verbatim}

  for a csh-style shell.

  When connecting to a remote machine using ssh, \envvar{DISPLAY} is
  normally set automatically (depending on how ssh has been
  configured).  However, this results in poor display performance
  because of encryption activity on the connection.  To increase
  performance you may override the value of \envvar{DISPLAY} provided
  by ssh.  Simply set \envvar{DISPLAY} as
  shown above.  Note that this technique defeats the encryption
  protection on the X11 connection.

  You will need to grant the remote machine permission to display on
  your local machine if you are using a telnet-like connection or if
  you are overriding the value of \envvar{DISPLAY} provided by ssh.
  Use the \command{xhost} command on the local machine to do
  this:

\begin{verbatim}
  xhost +remote-machine-name
\end{verbatim}

\envitem{LOAD\_PACKAGE}
  
  A comma seperated list of package names that \sr{} will load.
  
  This overrides the packages specified at configure time.  Note
  that this will not cause packages not specifed at configure time
  to magically compile and load though!  The package must already be
  compiled.  Whitespace is not tolerated after commas.
  
  Its value defaults to the list of packages given to the
  \option{\-\-enable-package} option at configure time.
  
\envitem{PACKAGE\_SRC\_PATH}
  
  A colon separated list of paths to root(s) of package source trees.
  The path \directory{\ptext{build\_dir}/Packages} is implicitly part
  of this list.  Note that \ptext{build\_dir} is the directory in
  which the \command{configure} and \command{make} commands were run.
  
  \sr{} searches this path list to find XML description files and
  user interface code corresponding to the list of packages
  specified in \envvar{LOAD\_PACKAGE}.
  
  It's value defaults to \directory{\ptext{build\_dir}/Packages}.

\envitem{PACKAGE\_LIB\_PATH}
  
  \envvar{PACKAGE\_LIB\_PATH} is a colon separated list of paths to
  package libraries.
  
  \sr{} searches this path list to find package libraries
  which correspond to the list of packages specified in
  \envvar{LOAD\_PACKAGE}.
  
  It default value is \directory{\ptext{build\_dir}/lib}.

  
\end{description}

\subsection{\sr{}'s Initialization File---\filename{.scirunrc}}
\label{sec:scirunrc}

One of the first thing \sr{} does when it starts up is to look for and
load the content of the file \filename{.scirunrc}.  \sr{} searches for
\filename{.scirunrc} in this order:

\begin{enumerate}
\item The current working directory.
\item The root of \sr's build directory.
\item Your home directory.
\item The root of \sr's source code directory.
\end{enumerate}

This file may contain assignment of values to variables and variable
substitution, for example

\begin{verbatim}
HOME=/home/sci/me
SCIRUN_ON_THE_FLY_LIBS_DIR=$(HOME)/on-the-fly-libs
\end{verbatim}

The expansion of environment variables is not supported.  Only
variables declared earlier in the file may be expanded.

The following variables are understood by \sr{} and may be set in
\filename{.scirunrc}: \envvar{SCIRUN\_ON\_THE\_FLY\_LIBS\_DIR},
\envvar{LOAD\_PACKAGE}, \envvar{PACKAGE\_SRC\_PATH},
\envvar{PACKAGE\_LIB\_PATH}.

\secref{Dynamic Compilation}{sec:dyncomp} explains the meaning and use
of \envvar{SCIRUN\_ON\_THE\_FLY\_LIBS\_DIR}.  See \hyperref{the
  previous section}{Section~}{}{sec:environ}
for the meanings of the other variables.

\subsection{Dynamic Compilation}
\label{sec:dyncomp}

Before executing \sr{} you should be aware of a feature called
\dfn{Dynamic Compilation}.

Dynamic compilition is a technique used by \sr{} that discovers and
generates code for the data types and algorithms used by modules in
your networks.  This is done at runtime and is done once for each new
data type and algorithm encountered.  This technique provides a number
of benefits not discussed here (see the publication 
\htmladdnormallinkfoot{\etitle{Dynamic
    Compilation of C++ Template Code}}{http://www.sci.utah.edu/publications/mcole01/dyn.pdf}
for details).

By default, code generated by dynamic compilation is stored in the
directory \directory{\ptext{build\_dir}/on-the-fly-libs} (where
\ptext{build\_dir} is the directory in which \sr{} was built).  The
location of dynamically generated code can be changed (and doing so
provides a number of benefits---see below) by setting the value of the
environment variable \envvar{SCI\_ON\_THE\_FLY\_LIBS\_DIR} to the
desired directory. For example:

\begin{verbatim}
SCIRUN_ON_THE_FLY_LIBS_DIR=$HOME/on-the-fly-libs
export SCIRUN_ON_THE_FLY_LIBS_DIR
\end{verbatim}

for Bourne-like shells (sh, ksh, bash, etc.) or

\begin{verbatim}
setenv SCIRUN_ON_THE_FLY_LIBS_DIR $HOME/on-the-fly-libs
\end{verbatim}

for csh-like shells.

It can also be changed by setting the value of this variable in
your \filename{.scirunrc}
file. Your \filename{.scirunrc} file would contain, for example:

\begin{verbatim}
SCIRUN_ON_THE_FLY_LIBS_DIR=/home/me/on-the-fly-libs
\end{verbatim}

Your \filename{.scirunrc} file is read by \sr{} at startup.

Customizing the value of this variable for each \sr{} user has
the following benefits:

\begin{itemize}
\item It allows multiple users to run the same instance of \sr{} (at
  the same time) without worrying about dynamic compilation conflicts.
  For example, by specifying an \filename{on-the-fly-libs}
  directory in their home directory, a user can run the same shared
  \sr{} installed in \directory{/usr/local} that another user
  is already running.

\item It allows a \directory{/usr/local} installation of \sr{}
  to be secure by not requiring that
  \directory{/usr/local/.../on-the-fly-libs} be writable by
  all (which is considered a potential security risk).

\item It allows greater debugging and multiple build support by
  allowing the user to change dynamically compiled code locations
  between instances of \sr{}.

\end{itemize}

Dynamic compilition will cause a delay the first time a module is
executed.  The module will turn a different color while it is being
compiled.

\subsection{Exiting \sr{}}
\label{sec:stopping}

Exit \sr{} by selecting the \menuitem{Quit} item from the \menu{File} menu

Don't press \keyboard{control-c} to exit \sr{}.  Doing this will drop
you into a debugger which is probably not what you want to do.


%%% Local Variables: 
%%% mode: latex
%%% TeX-master: "usersguide"
%%% End: 
