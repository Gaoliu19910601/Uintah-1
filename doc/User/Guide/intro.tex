% -*-latex-*-
%
%  The contents of this file are subject to the University of Utah Public
%  License (the "License"); you may not use this file except in compliance
%  with the License.
%
%  Software distributed under the License is distributed on an "AS IS"
%  basis, WITHOUT WARRANTY OF ANY KIND, either express or implied. See the
%  License for the specific language governing rights and limitations under
%  the License.
%
%  The Original Source Code is SCIRun, released March 12, 2001.
%
%  The Original Source Code was developed by the University of Utah.
%  Portions created by UNIVERSITY are Copyright (C) 2001, 1994
%  University of Utah. All Rights Reserved.
%

% intro.tex
%

\section{Introduction}
\label{sec:intro}


This is the \etitle{\srug}.  It describes the purpose and use, of the \sr{}
problem solving environment.  It is for those users who will be building
and executing \dfn{networks} within the \sr{} environment.

Those who will be installing \sr{} should read the
\htmladdnormallinkfoot{\srig{}}{\installguideurl}.


%\subsection{Conventions}
%\label{sec:conventions}

%\missing{Discussion of typographic conventions}

\subsection{Road Map}
\label{sec:roadmap}

This document is organized into the following main sections:

\begin{description}
  \item \secref{Introduction}{sec:intro} is what you are reading now.
  \item \secref{Concepts}{sec:concepts} introduces the concept of an
        integrated problem solving environment  and describes how \SR{}
        embodies some of these ideas.
  \item \secref{Starting \sr}{sec:startingup}  outlines the procedure for
        starting \sr{} and related information.
  \item \secref{Working with Networks}{sec:workwithnets} discusses tasks
        involved in building, editing, and executing networks.
  \item \secref{Visualization with the \viewer{}}{sec:viewer} describes the
        purpose and use of a visualization module called the \viewer{}.  The
        \viewer{} is probably \sr{}'s most commonly used module.
  \item \secref{Packages}{sec:packages} offers an overview of the content
        of the \sr{} and \pse{} packages (which provide nearly all of
        \sr{}'s computational services).
  \item \secref{Importing data into \sr{}}{sec:import} summarize ways to
        get data in your local format into and out of \SR{}.
\end{description}

\subsection{Getting Help}
\label{sec:help}

Help is available from several sources.

\subsubsection{In the Distribution}

This document and others related to \sr{} can be found in your \sr{}
distribution.  Point your browser at \filename{index.html} which is in your
distribution's top level \directory{doc} directory (\ie{} \ab{top of
distribution}/doc/index.html).  Latex and postscript versions of some
documents are also in the distribution's \directory{doc} directory.

\subsubsection{On the Web}

This and other documents related to \sr{} can be found in the
\htmladdnormallinkfoot{document}{\scidocurl{}} section of the
\htmladdnormallinkfoot{\sci{}}{\sciurl{}} web site.

Be sure to visit the main \sci{} web site for lots of other
information related to \sr{} and the \scii{}.

\subsubsection{From the Mailing Lists}

The \sr{} users mail list is a forum for discussing \sr{} related issues.
To subscribe send mail to:

\mailto{Majordomo@sci.utah.edu}

with the following command in the body of your message:

\keyboard{subscribe scirun-users}

\subsection{Reporting Bugs}
\label{sec:bugs}

Please report bugs!  To report a bug visit \sr{}'s
\htmladdnormallinkfoot{bug database}{\bugsurl} web page.

Reporting bugs this way (rather than by way of the mailing list) ensures
that they will be fixed in timely manner.



%%% Local Variables: 
%%% mode: latex
%%% TeX-master: "usersguide"
%%% End: 
