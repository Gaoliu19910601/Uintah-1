% -*-latex-*-
%
%  The contents of this file are subject to the University of Utah Public
%  License (the "License"); you may not use this file except in compliance
%  with the License.
%
%  Software distributed under the License is distributed on an "AS IS"
%  basis, WITHOUT WARRANTY OF ANY KIND, either express or implied. See the
%  License for the specific language governing rights and limitations under
%  the License.
%
%  The Original Source Code is SCIRun, released March 12, 2001.
%
%  The Original Source Code was developed by the University of Utah.
%  Portions created by UNIVERSITY are Copyright (C) 2001, 1994
%  University of Utah. All Rights Reserved.
%

% network.tex
%

\section{Working with Networks}
\label{sec:workwithnets}

This section describes how to create, save, load, execute, and edit
networks.

When started with no arguments the \command{scirun} command will create a
main window with a blank NetEdit frame.  A user creates and
connects modules to form a network.


\subsection{Creating a Module}
\label{sec:creatingmodules}

To create a module select its name from one of the package (\eg{} \sr)
menus' category sub-menus.  The package menus may be accessed from the
main window's menu bar and from the NetEdit frame's pop-up menu. You
may activate the NetEdit frame's pop-up menu by clicking mouse button
3 while the mouse pointer is in the NetEdit frame (but not over a
module or connection).  The pop-up menu contains a list of category
sub-menus from the \sr{} package and other installed packages.  And
each category sub-menu provides access to the modules within the
category.

After creating a module its graphic representation will be created and
placed in the NetEdit frame.

\subsection{Anatomy of a Module}
\label{sec:modanatomy}

%begin{latexonly}
  \newcommand{\modgraphic}%
  {\centerline{\epsfig{file=Figures/modgraphic-1.eps.gz,width=4in,
        bbllx=0, bblly=0, bburx=325, bbury=157}}}
%end{latexonly}
\begin{htmlonly}
  \newcommand{\modgraphic}{%
  \htmladdimg[align=top,width="256",alt="SCIRun Module Graphic"]
  {../Figures/modgraphic-1.gif}}
\end{htmlonly}

All modules are represented similarly by a graphic within the NetEdit frame.
See Figure~\ref{fig:modgraphic}. This graphical ``front end'' is the same
for all modules and consists of the following elements:

\begin{figure}[htb]
  \begin{makeimage}
  \end{makeimage}
  \modgraphic
  \caption{\label{fig:modgraphic} Module Graphic (\module{Show
      Field} Module)}
\end{figure}

\begin{description}
  \descitem{Module Name} The module's name.
  
  \descitem{Input Ports} Zero or more input ports located on the top
  of the module.  Each port corresponds to a data type and each data
  type has a unique color.  Table~\ref{tab:portcolors} maps port
  colors to data types.  Input ports connect to other modules' output
  ports.  Connections can only be made between ports of the same type.

  \begin{table}[htbp]
    \begin{center}
      \begin{tabular}{|l|l|}
        \hline
        \textbf{Data Type} & \textbf{Port Color} \\
        \hline
        Field & Yellow \\
        Field Set & Green \\
        Matrices & Blue \\
        Geometric Objects & Pink \\
        Color Maps & Purple \\
        Camera Path & Brown \\
        \hline
      \end{tabular}
      \caption{Data Types and their Port Colors}
      \label{tab:portcolors}
    \end{center}
  \end{table}
  
  \descitem{Output ports} Zero or more output ports located on the
  bottom of the module.  Output ports connect to other modules' input
  ports.  Every module has, of course, at least 1 input or 1 output
  port.
  
  \descitem{UI button} Pressing the \button{UI} button displays the
  module's control dialog. Some modules have no such dialog. Some have
  very simple dialogs.  Some have very complex dialogs which allow
  elaborate control over the module.  Figure~\ref{fig:moddialog} shows
  the control dialog for \module{Show Field} module.
  
  \descitem{Progress bar} Shows the module's progress.  As the module
  works towards completion of its task, the Progress bar is filled
  with first red, then yellow, and then green.  When the Progress bar
  is all green the module is done.
  
  \descitem{Timer} Displays the amount of CPU time the module has
  consumed.  Located to the left of the progress bar.
  
  \descitem{Message Indicator} Shows the presence of messages in a
  module's log.  Colors represent message types.  Blue is for
  ``remarks'' (informational messages) and yellow is for ``warnings''
  (your attention is needed!).  Click the indicator to display the
  module's log.

\descitem{Pop-up Menu} Pressing mouse button 3 while the mouse
  pointer is over a module gives access to the module's pop-up menu.  The
  pop-up menu gives access to the following items:
  \begin{description}
    \menuitemdesc{::Package\_Category\_Name\_Instance} This item is a
    label (not a selectable item).  It provides the module's name and
    the category and package to which to module belongs.  The Instance
    part is a unique number which distinguishes multiple instances of
    the same module.

    \menuitemdesc{Execute} Tells the module to
    execute (or re-execute).  Depending on the state of its input and
    output ports this will cause other modules in the network to
    ``fire'' (execute) also..

    \menuitemdesc{Help} Displays the module's help window.
    
    \menuitemdesc{Notes} This item displays the module's note pad.
    You can use the note pad to document the purpose of the module in
    the current network.

    \menuitemdesc{Destroy} Destroys the module.
    
    \menuitemdesc{Show Log} Displays the module's message log.  Most
    modules will write messages to their log during the course of
    their execution.
    
    \menuitemdesc{Show Status} This item is a toggle button which
    turns off/on the display of the progress indicator.  Turning off
    the progress indicator can help speed up the execution of complex
    networks.
  \end{description}
\end{description}


\subsection{Creating a Connection}
\label{sec:connectmods}

Mouse button 2 (the middle mouse button) is used to connect the output
(input) port of one module to the input (output) port of another
module.

To make a connection, position the mouse pointer over a module's input
(output) port.  Then press button 2 and drag the the mouse pointer towards
another module's output (input) port.

When button 2 is first pressed the program shows all valid connections as
black lines.  It also draws one red colored connection which is the
connection that will be made if you stop the drag by releasing 
button 2.

Make the connection by releasing button 2 when the pointer is over
the desired destination port or when the red colored connection is the
desired one.  The connection will be drawn using the color
corresponding to the connection's data type.

You may connect a module's output port to the input ports of 1 or more
modules by repeating the procedure just described.

\subsection{Setting Module Properties}
\label{sec:setmodprops}

%begin{latexonly}
  \newcommand{\moddialog}%
  {\centerline{\epsfig{file=Figures/moddialog.eps.gz,
        bbllx=0, bblly=0, bburx=272, bbury=406}}}
%end{latexonly}
\begin{htmlonly}
  \newcommand{\moddialog}{%
  \htmladdimg[align=top,alt="SCIRun Module Dialog"]
  {../Figures/moddialog.gif}}
\end{htmlonly}

To change a module's properties click its \button{UI} button.  This will
display the module's control dialog.  Use this dialog to change the
module's properties.  Each module's reference documentation explains the
use of its control dialog.  Figure~\ref{fig:moddialog} shows the control
dialog for the \module{Show Field} module.

\begin{figure}[htb]
  \begin{makeimage}
  \end{makeimage}
  \moddialog
  \caption{\label{fig:moddialog} Module \module{Show Field}'s Control Dialog
    (User Interface).}
\end{figure}

\subsection{Executing a Network}
\label{sec:executenet}

``Network Execution'' means that 1 or more modules must be executed in a
coordinated fashion. The coordinated execution of modules is managed by
\sr{}'s \dfn{scheduler}.

Note that some modules may need to be compiled before they are
executed (see \secref{Dynamic Compilation}{sec:dyncomp}).  Compilation
will delay execution of the network.  This delay occurs only one time
(per module).  After a module has been compiled it need not be
compiled again.  The modules will turn a different color while they
are being compiled.


\subsubsection{The Basics}

The scheduler is invoked when an event occurs that \dfn{triggers} a
module's execution.  The scheduler creates a list of all modules that
must be executed in coordination with the triggered module. Modules
\dfn{upstream} (directly or indirectly) from the triggered module will
be put on the execution list if they have not previously executed.
All modules \dfn{downstream} from the triggered module will be put
on the list.  Once the scheduler determines which modules must be
executed, it executes them (in parallel where possible).

Network execution is mostly transparent.  That is, the events that trigger
module execution will usually be generated automatically while you work
(\eg by changing a module's property).  Sometimes though you must manually
generate a triggering event by choosing the \menuitem{Execute} item from a
module's pop-up menu.

\subsubsection{Details}

Each module executes in its own thread and blocks (waits) until its upstream
modules can supply it with data.  After a module completes its computation
it sends its results to its downstream modules.  This completes a module's
execution cycle.  Its next chance to receive data from its input ports
and send data to its output ports will not arise until some event
causes it to be put on the scheduler's execution list again.  

This behavior prevents modules from computing in an iterative fashion,
sending intermediate results to their downstream modules.  This is because
downstream modules cannot receive these results until they are in their
execution cycle.  They would need to be executed each time the
upstream module posts an intermediate result.


\subsubsection{Intermediate Results}

Some modules are designed to be used in an iterative fashion.  They send
data to their output ports in a special way.  They use a method called
\icode{send\_intermediate} to send the results of each iteration.  When
this method is used the scheduler (re)executes downstream modules each time
the upstream module posts its next result.  Downstream modules are
therefore able to receive the results of each iteration as soon as the
upstream module sends them.

Modules \module{SolveMatrix} and \module{MatrixSelectVector} (from the
\package{\sr} package and \category{Math} category) are examples of modules
that compute iteratively using the \icode{send\_intermediate} method.

\subsubsection{Feedback Loops}

Some modules are designed to be used in feedback loops (and if fact,
they can only be used this way).  Their output ports may be connected
directly or indirectly to their input ports.  These modules also use the
\icode{send\_intermediate} method.

Examples of feedback modules are \module{DipoleSearch} and
\module{ConductivitySearch} from the \category{Inverse} category of the
\package{BioPSE} package and \module{BuildElemLeadField} from the
\category{LeadField} category of the \package{BioPSE} package.

\subsection{Deleting a Connection}
\label{sec:deleteconnections}

Delete a connection by pressing button 3 while the pointer is
over a connection.

\subsection{Selecting Modules}
\label{sec:selectmods}

A few operations (see \secref{Moving Module(s)}{sec:movemod} and
\secref{Destroying Module(s)}{sec:destroymod}) act on a group of
modules.  There are 2 ways to create a group of modules.

\begin{enumerate}
\item Press mouse button 2 while the pointer is in the NetEdit frame but not over
a module or a connection.  Then drag the mouse.  This will draw a box
outline.  Modules intersecting the box will be part of the group.  Release
button 2 to complete your selection.
\item Select the first module by clicking mouse button 2 while the pointer is
over a module.  Then add modules to the group by pressing (and holding
down) the \keyboard{control} key while you click on modules with button 2.
\end{enumerate}

Selected modules will be drawn in a slightly darker shade of grey.

You may mix selection methods---you may always add modules to the group by
pressing the control key while making selections using the above methods.
If you don't press the control key then the previous group will be
forgotten and a new one will be created.

\subsection{Moving Module(s)}
\label{sec:movemod}

Modules may be moved around in the NetEdit frame.  To move a module simply
press button 1 while the pointer is over a module and drag the module to
its new location.

Multiple modules may be moved at one time.  To do this first
\hyperref{select}{select (see Section~}{)}{sec:selectmods} 1 or more
modules. Then press button 1 while the pointer is over any one of the
modules in the selected group and drag the modules to their new location.


\subsection{Destroying Module(s)}
\label{sec:destroymod}

Delete a module by selecting the \menuitem{Destroy} menu item from the
module's pop-up menu.

Multiple modules may be deleted at one time.  To do this first
\hyperref{select}{select (see Section~}{)}{sec:selectmods} 1 or more
modules. Then choose, from the pop-up menu of any one of the modules in the
selected group, the \menuitem{Destroy Selected} item.


\subsection{Documenting a Module's Use}
\label{sec:docmodule}

It is often useful to document the purpose of a module within a network.
Each module maintains a note pad for this purpose.  You may edit the
module's note pad by selecting \menuitem{Notes} from the module's
pop-up menu.  This will display the module's note pad editor.  This 
editor allows you to write notes on the use of the
module within the context of its network.

\subsection{Viewing a Module's Log}
\label{sec:viewmodslog}

Each module supports a message log.  The module may write error messages or
other types of messages to its log.  To view this log select the
\menuitem{Show Log} item from the module's pop-up menu.

\subsection{Documenting a Network}
\label{sec:}

It is useful to document the purpose of a network.  You may use a network's
note pad for this purpose.  To edit the network's note pad select the
\menuitem{Add Info} item from the main window's \menu{File} menu.  This
will display the network's note pad editor.  This editor allows you
to write notes on the purpose and use of the network.


\subsection{Saving a Network}
\label{sec:savenet}

\sr{} can save networks to files.  Network files have an extension of
\filename{.net} (although in the past they have also had .sr and .uin
extensions).  

To save a network select the \menuitem{Save} item from the main window's
\menu{File} menu.  A file browser dialog will prompt you for the
name (and location) of the network file.

If you make changes to an existing network then \menuitem{Save} will
save your changes to the existing net file.

You may save an existing network under a new name using the
\menuitem{Save As...} menu item.  A file browser dialog will prompt
you for the new name of the network file.  Subsequent uses of
\menuitem{Save} will save changes to the newly created file.

Network files are actually \dfna{Tool Command Language}{TCL} scripts.
These files may be edited although the reasons for doing so are beyond
the scope of this guide.

\subsection{Loading a Network}
\label{sec:opennet}

To load a network file select the \menuitem{Load} item from the main
window's \menu{File} menu.   A file browser dialog will prompt you for the
name (and location) of the network file.

Note that loading a network file adds the network to an existing network in
the NetEdit frame possibly overlapping the networks.

\subsection{Inserting a Network}
\label{sec:insertnetwork}

In order to avoid merging networks, the user may select the
\menuitem{Insert} item from the main window's \menu{File} menu. This
option allows the user to place one \sr{} network next to another,
avoiding overlap.  A file browser dialog prompts the user for the name (and
location) of the network file.

The new network will be inserted into the upper left corner of the NetEdit
frame.  If a network of modules already exists in the NetEdit frame, the
Insert option places a new network to the immediate right of that existing
network. 

\subsection{Clearing a Network}
\label{sec:clearnetwork}

Select the \menuitem{Clear} item from the main window's \menu{File}
menu in order to remove all modules and connections from the entire
NetEdit frame.  A text box appears, confirming whether the user wishes
to proceed with or cancel the clearing operation.

\subsection{Navigating a Network}
\label{sec:navnetwork}

A complex network may not be entirely visible in the NetEdit frame.  You may
use the NetEdit frame's scroll bars to bring other parts of a network into
view.  Or you may use the navigation tool to view complex networks.

The Global Frame shows the entire ``network world.''  The small
rectangular region (outlined in black) within the  is the
network view tool and it is a window on the network world.  The
position of the view tool determines the part of the network that is
visible in the NetEdit frame.  To view other parts of the network, press
button 1 while the pointer is anywhere in the Global View Frame -- this will
move the tool to the location of the pointer --  then drag the tool to the
new location.


\subsection{The \sr{} Shell}
\label{sec:termapp}

After starting, \sr{} will run a shell-like application in the terminal
window called the \dfn{\sr{} shell}.  It displays the prompt
\screen{scirun\ra} in the terminal window.  This program is a
\dfna{Tool Command Language}{TCL} shell program that has been extended with
\sr{} specific commands.

It possible to type \tcl{} \sr{} commands at the prompt.  For
instance, to load a network file you may type \keyboard{source
  \ptext{network file name}}.  This has the same effect as the \menu{File}
menu's \menuitem{Load} command.

%\incomplete{}


%%% Local Variables: 
%%% mode: latex
%%% TeX-master: t
%%% End: 
