%
%  For more information, please see: http://software.sci.utah.edu
% 
%  The MIT License
% 
%  Copyright (c) 2004 Scientific Computing and Imaging Institute,
%  University of Utah.
% 
%  
%  Permission is hereby granted, free of charge, to any person obtaining a
%  copy of this software and associated documentation files (the "Software"),
%  to deal in the Software without restriction, including without limitation
%  the rights to use, copy, modify, merge, publish, distribute, sublicense,
%  and/or sell copies of the Software, and to permit persons to whom the
%  Software is furnished to do so, subject to the following conditions:
% 
%  The above copyright notice and this permission notice shall be included
%  in all copies or substantial portions of the Software.
% 
%  THE SOFTWARE IS PROVIDED "AS IS", WITHOUT WARRANTY OF ANY KIND, EXPRESS
%  OR IMPLIED, INCLUDING BUT NOT LIMITED TO THE WARRANTIES OF MERCHANTABILITY,
%  FITNESS FOR A PARTICULAR PURPOSE AND NONINFRINGEMENT. IN NO EVENT SHALL
%  THE AUTHORS OR COPYRIGHT HOLDERS BE LIABLE FOR ANY CLAIM, DAMAGES OR OTHER
%  LIABILITY, WHETHER IN AN ACTION OF CONTRACT, TORT OR OTHERWISE, ARISING
%  FROM, OUT OF OR IN CONNECTION WITH THE SOFTWARE OR THE USE OR OTHER
%  DEALINGS IN THE SOFTWARE.
%



\chapter{Wrapping  ITK Filters in \sr{} Modules}
\label{ch:itk_mods}
\index{Insight Toolkit}

The \htmladdnormallinkfoot{National Library of Medicine Insight
  Segmentation and Registration Toolkit}{http://www.itk.org} (ITK)
is open source software that complements \sr{} by providing
image-processing capabilities ranging from fundamental algorithms to
advanced segmentation and registration tools.  The Insight Tool Kit
provides these capabilities as \dfn{filters}.  \sci{} has developed a
method of wrapping ITK filters in \sr{} modules.

If ITK is present, and \sr{} is appropriately configured, \sr{} 
builds a default set of ITK filter-based modules.  Users can generate 
additional modules by writing filter descriptions using \sr{}'s
XML-based filter description language. This chapter describes
the process of wrapping ITK filters in \sr{} modules.

Before reading this chapter, users should be familiar with \sr{}, the
Insight toolkit, and the use of generic programming techniques.  Users
must have access to the \sr{} source code tree and be familiar with
its organization.


\section{Aim}
\label{sec:itk_mods:aim}

ITK filters are C++ objects (also known as \dfn{process objects}) that
receive data on input ports and forward results on output ports.
ITK filter objects provide functions for establishing connections to
ports and setting filter properties.

\sr{} modules are C++ objects that receive data via input ports
and forward results via output ports.  \sr{} modules provide
GUI-based dialogs for modifying  module properties.

The aim of \sr{}/ITK integration is to map ITK filter data types,
ports, properties, and property editing functions to \sr{} module data
types, ports, properties, and property editing GUI dialogs.


\section{Approach}
\label{sec:itk_mods:approach}

ITK filters are mapped onto \sr{} modules via XML description files.
Up to three XML files are required for each filter:

\begin{description}
\descitem{ITK Filter Description File} Description of a filter.  Contains
no \sr{} specific information.  
\descitem{\sr{} Filter Description File} Contains \sr{} specific
information and include a reference to an ITK filter description file
and an optional reference to a GUI description file.
\descitem{GUI Description File} This file is optional.  It describes a
filter's GUI.  A default GUI will be create if this file does not exist.
\end{description}

Software in the \sr{} build system automatically translates these XML
files to \sr{} modules.   XML files are described in
\secref{ITK Filter Description}{sec:itk_mods:itk_filter_desc},
\secref{\sr{} Filter Description}{sec:itk_mods:sr_filter_desc}, and
\secref{\sr{} GUI Description}{sec:itk_mods:sr_gui_desc},
respectively.


\section{ITK Filter Description}
\label{sec:itk_mods:itk_filter_desc}

The ITK filter description XML file describes a filter's inputs,
outputs, parameters, parameter modifying member functions, templated
data types, and other filter information.  This file contains no \sr{}
specific information.  

ITK filter description files are located in
directory \directory{SCIRun/src/Packages/Insight/ITK}. 
ITK filter description files follow the naming convention:

\begin{alltt}
  itk\_\replaceable{filter_name}.xml
\end{alltt}

for example:

\begin{alltt}
  itk\_BinaryThresholdImageFilter.xml
\end{alltt}


The following XML code illustrates the overall content of an ITK
filter description file:

\begin{alltt}
  <?xml version="1.0" encoding="iso-8859-1"?>
  <!DOCTYPE filter SYSTEM "itk_filter.dtd">
  <filter-itk name="\replaceable{filter_name}">
     <description>
     \velide
     </description>
     <templated>
     \velide
     </templated>
     <datatypes>
     \velide
     </datatypes>
     <inputs>
     \velide
     </inputs>
     <outputs>
     \velide
     </outputs>
     <parameters>
     \velide
     </parameters>
     <includes>
     \velide   
     </includes>
  </filter-itk>
\end{alltt}

Each element is discussed below.

\subsection{Element \xmlstarttag{filter-itk}}
\label{sec:itk_mods:filter-itk_element}

Element \xmlstarttag{filter-itk} is the top-level element.  All other
elements are enclosed in element \xmlstarttag{filter-itk}. The value
of the required ``name'' attribute is the qualified C++ filter name.

For example:

\begin{alltt}
  <filter-itk name="itk::ReflectImageFilter">
    \velide
  </filter>
\end{alltt}

\subsection{Element \xmlstarttag{description}}
\label{sec:itk_mods:descr_element}

Element \xmlstarttag{description} should provide a short (several sentences)
filter description.  

For example:

\begin{alltt}
  <description>
    Implements a Reflection of an image along a selected 
    direction. This class is parameterized over the type 
    of the input image and the type of the output image. 
  </description>
\end{alltt}


\subsection{Element \xmlstarttag{templated}}
\label{sec:itk_mods:templated}

ITK filters are templated (parameterized) C++ classes and generally
support one or two template parameters.  The \xmlstarttag{templated}
element describes a filter's template information.  Specifically, 
\xmlstarttag{templated} contains elements that name a filter's
template parameters and list suitable data types for filter
instantiations.

\begin{alltt}
<templated>
  <template>\replaceable{template\_parameter\_name1}</template>
  <template>\replaceable{template\_parameter\_name2}</template>
  \velide 
  <defaults>
    <default>\replaceable{data\_type\_for\_parameter\_name1}</default>
    <default>\replaceable{data\_type\_for\_parameter\_name2}</default>
    \velide
  </defaults>
  <defaults>
    <default>\replaceable{data\_type\_for\_parameter\_name1}</default>
    <default>\replaceable{data\_type\_for\_parameter\_name2}</default>
    \velide
  </defaults>
  \velide
</templated>
\end{alltt}

Each \xmlstarttag{template} element declares and names a template
parameter.  Template parameter names may be used in subsequent
\xmlstarttag{input} and \xmlstarttag{output} elements---see below.

Each \xmlstarttag{defaults} element corresponds to one filter class
instantiation.  Each \xmlstarttag{default} element names a type that
corresponds (in order) with previous \xmlstarttag{template} elements.

For example, the following \xmlstarttag{templated} element:

\begin{alltt}
  <templated> 
    <template>InputImageType</template> 
    <template>OutputImageType</template>
    <defaults>
       <default>itk::Image&lt;float, 2&gt;</default> 
       <default>itk::Image&lt;float, 2&gt;</default> 
    </defaults> 
  </templated>
\end{alltt}

corresponds to filter class template:

\begin{alltt}
  template <class InputImageType, class OutputImageType>
  class \replaceable{AnITKFilterClass}\(\ldots\)
\end{alltt}

and filter type instantiation:

\begin{alltt}
  \replaceable{AnITKFilterClass}<itk::Image<float,2>,itk::Image<float,2> >
\end{alltt}

Note that characters ``\la'' and ``\ra'' cannot be typed literally into an
XML file. The character entities '\&lt;' and '\&gt;', respectively, 
must be used.

A \xmlstarttag{template} element and its corresponding
\xmlstarttag{default} element can describe a parameterized built-in 
value.  In this case, a \xmlstarttag{template} element's ``type''
attribute specifies a C++ built-in type, and a corresponding
\xmlstarttag{default} element specifies the type's value:

\begin{alltt}
  <templated>
    <template type="unsigned int">ImageDimension</template>
    \velide
    <defaults>
      <default>2</default>
      \velide
    </defaults>
  </templated>
\end{alltt}

A \sr{} filter module knows all possible filter instantiations.  When
executed at runtime, a module uses the filter instantiation with data
input and output types matching data types passing through the
module's input and output ports.

\subsection{Element \xmlstarttag{datatypes}}

Element \xmlstarttag{datatypes} declares filter parameter data types
that are not C++ built-in  types.  Names of declared data types are
used in \xmlstarttag{parameter} elements that describe filter
parameters (see \secref{Element).
  \xmlstarttag{parameters}}{sec:itk_mods:param_element}).  Element
\xmlstarttag{datatypes} is optional, it is needed only when a
parameter is not a C++ built-in type.

The structure of the \xmlstarttag{datatypes} element follows:

\begin{alltt}
  <datatypes>
    \xmlstarttag{\replaceable{data\_type\_element} name="\replaceable{data\_type\_name}"}
    \velide
    \xmlendtag{\replaceable{data\_type\_element}}
    \velide  
  </datatypes>
\end{alltt}

where \replaceable{data\_type\_element} is the name of a \dfn{data
  type element}.  

Each \xmlstarttag{\replaceable{data\_type\_element}}
describes the properties of one data type.  Element
\xmlstarttag{datatypes} enclose one or more data type elements.

At this time, element \xmlstarttag{array} is the only supported data
type element.  The array-like type described by element
\xmlstarttag{array} is subject to the following conditions:

\begin{itemize}
\item It must support operator[]()
\item Its array elements must be a built-in type
\item It must support a member function that returns the array length
\end{itemize}

The structure of element \xmlstarttag{array} follows:

\begin{alltt}
  <array name="FilterType::\replaceable{data\_type\_name}">
    <elem-type>\replaceable{built-in\_type}</elem-type>
    <size-call>\replaceable{member\_function}</size-call>
  </array>
\end{alltt}

Attribute \xmlattrname{name} specifies the name of an array-like
data type.  ``FilterType'' is a pseudo qualifier.  It represents the
full templated type of the current filter.  \datatype{FilterType} is
used in place of the full template name, for example, 
\datatype{itk::MeanImageFilter<InputType,OutputType>}.
\replaceable{Data\_type\_name} is the name of a type declared or
typedef'ed in the filter's class declaration.

Element \xmlstarttag{elem-type} specifies the array's element type.
\replaceable{Built-in\_type} is a C++ built-in type.

Element \xmlstarttag{size-call} specifies the name of a member
function that returns the number of elements in the array.

An example \xmlstarttag{datatypes} element with an \xmlstarttag{array}
sub-element follows:

\begin{alltt}
  <datatypes>
    <array name="FilterType::InputSizeType">
      <elem-type>int</elem-type>
      <size-call>GetSizeDimension</size-call>
    </array>
  </datatypes>
\end{alltt}

\datatype{FilterType::InputSizeType} can be used in a subsequent
\xmlstarttag{param} element.


\subsection{Element \xmlstarttag{inputs}}
\label{sec:itk_mods:inputs_element}

The \xmlstarttag{inputs} element describes data types accepted by a
filter's input ports, and member functions used to
establish input port connections:

\begin{alltt}
  <inputs>
    <input name="\replaceable{input\_name}">
      <type>\replaceable{data\_type\_name}</type>
      <call>\replaceable{member\_function}</call>
    </input>
    \velide
  </inputs>
\end{alltt}

For example:

\begin{alltt}
  <inputs>
    <input name="SourceImage">
      <type>InputImageType</type>
      <call>SetInput</call>
    </input>
  </inputs>
\end{alltt}

Element \xmlstarttag{input} declares one filter input.
The \xmlattrname{name} attribute names an input.  The value of
\xmlattrname{name} names the corresponding \sr{} module input port.
The value of \xmlattrname{name} must be unique among all inputs.

Element \xmlstarttag{type} specifies the data type accepted by a
filter's input port.  \replaceable{Data\_type\_name} can be a name
declared in a \xmlstarttag{template} element (as shown above), or a
parameterized data type with  parameter names declared in a
\xmlstarttag{template} element. For example, the following template
element associates type \datatype{float} with name
\datatype{ScalarType}:

\begin{alltt}
  <templates>
    <template>ScalarType</template>
    <defaults>
      <default>float</default>
    </defaults>
  </templates>
\end{alltt}

and \datatype{ScalarType} is used as a template parameter in the
following \xmlstarttag{inputs} element:

\begin{alltt}
  <inputs>
    <input name="InputImage">
      <type>itk::watershed::SegmentTree&lt;ScalarType&gt;</type>
      <call>SetInputSegmentTree</call>
    </input>
  </inputs>
\end{alltt}

Element \xmlstarttag{call} specifies a member function (commonly
\function{SetInput}) that establishes an input port connection.

\subsection{Element \xmlstarttag{outputs}}
\label{sec:itk_mods:outputs_element}

The \xmlstarttag{outputs} element describes  data types transmitted
on a filter's output ports, and the member functions used to
establish  output port connections:

\begin{alltt}
  <outputs>
    <output name="\replaceable{output\_name}">
      <type>\replaceable{data\_type\_name}</type>
      <call>\replaceable{member\_function}</call>
    </output>
    \velide
  </outputs>
\end{alltt}

For example:

\begin{alltt}
  <outputs>
    <output name="DestImage">
      <type>OutputImageType</type>
      <call>GetOutput</call>
    </output>
  </outputs>
\end{alltt}


The \xmlattrname{name} attribute names an output.  The value of
\xmlattrname{name} names the corresponding \sr{} module output port.
The value of \xmlattrname{name} must be unique among all outputs.

Element \xmlstarttag{type} specifies the data type transmitted on a
filter's output port.  \replaceable{Data\_type\_name} can be a name
declared in a \xmlstarttag{template} element (as shown above), or a
parameterized data type with parameter names declared in a
\xmlstarttag{template} element.

Element \xmlstarttag{call} specifies the member function (commonly
\function{GetOutput}) that establishes an output port connection.  If
\xmlstarttag{call} element's value is a member function returning a
constant pointer, \xmlstarttag{call}'s \xmlattrname{const} attribute
must be set to the value ``yes'', for example:

\begin{alltt}
  <call const="yes">GetSpeedImage</call>
\end{alltt}

\subsection{Element \xmlstarttag{parameters}}
\label{sec:itk_mods:param_element}

A filter's parameters determine its behavior.  Filter member functions
change parameter values.  The \xmlstarttag{parameters} element
describes a filter's parameters:

\begin{alltt}
  <parameters>
    <param>
      <name>\replaceable{parameter\_name}</name>
      <type>\replaceable{data\_type}</type>
      <call>\replaceable{member\_function}</call>
      <default>\replaceable{initial\_value}</default>
    </param>
    \velide  
  </parameters>
\end{alltt}

Each \xmlstarttag{param} element describes one parameter with
sub-elements \xmlstarttag{name}, \xmlstarttag{type},
\xmlstarttag{call}, and optional \xmlstarttag{default} providing the
parameter's name, data type, corresponding member function, and
optional initial value.

For example:

\begin{alltt}
  <parameters>
    <param>
      <name>upper\_threshold</name>
      <type>float</type>
      <call>SetUpperThreshold</call>
      <default>1000.0</default>
    </param>
  </parameters>
\end{alltt}

The content of element \xmlstarttag{type} is either a C++ built-in
type or type declared in a \xmlstarttag{datatypes} element.  Element
\xmlstarttag{default} is optional.

If the ITK filter class macro \icode{itkBooleanMacro} is used to create a
boolean parameter, two \xmlstarttag{call} elements are needed; one
for setting a true value, and one for setting a false value:

\begin{alltt}
  <call value="on">\replaceable{member\_function\_on}</call>
  <call value="off">\replaceable{member\_function\_off}</call>
\end{alltt}

For example:

\begin{alltt}
  <param>
    <name>reverse\_expansion\_direction</name>
    <type>bool</type>
    <call value="on">ReverseExpansionDirectionOn</call>
    <call value="off">ReverseExpansionDirectionOff</call>
  </param>
\end{alltt}


To modify the parameter, each parameter requires a GUI. If a GUI
description file is not created, a default GUI is generated that
associates a text entry widget with each filter parameter.
\secref{\sr{} GUI Description}{sec:itk_mods:sr_gui_desc} illustrates 
writing a GUI description for a filter module.  When writing a GUI
description file, note that the value of the \xmlstarttag{name}
element above must match the value of a corresponding
\xmlstarttag{param} element's \xmlattrname{name} attribute in the GUI
description file.

\subsection{Element \xmlstarttag{includes}}
\label{sec:itk_mods:includes_element}

Element \xmlstarttag{includes} lists ITK include files declaring filter
classes and data types:

\begin{alltt}
<includes>
  <file>\replaceable{include\_file}<file>
</includes>
\end{alltt}

For example:

\begin{alltt}
<includes>
  <file>itkBinaryThreasholdImageFilter.h<file>
</includes>
\end{alltt}


\section{\sr{} Filter Description}
\label{sec:itk_mods:sr_filter_desc}

A \sr{} filter description file is associated with each ITK filter.
The \sr{} filter description references the ITK filter description
file, the optional GUI file, and contains \sr{} specific information.

\sr{} filter description files are located in directory:

\begin{alltt}
  SCIRun/src/Packages/Insight/Dataflow/Modules/Filters/XML
\end{alltt}

\sr{} filter description files follow the naming convention:

\begin{alltt}
  sci\_\replaceable{filter_name}.xml
\end{alltt}

for example:

\begin{alltt}
  sci\_BinaryThresholdImageFilter.xml
\end{alltt}


The following XML code illustrates the overall structure of a
sci-filter description file:

\begin{alltt}
  <?xml version="1.0" encoding="iso-8859-1"?>
  <!DOCTYPE filter SYSTEM "sci_filter.dtd">
  <filter>
    <include href="\replaceable{file_path}"/>
    \velide
    <filter-sci name="\replaceable{\sr{} module name}">
      <package>\replaceable{SCIRun\_package\_name}</package>
      <category>\replaceable{SCIRun\_category\_name}</category>
      <instantiations>
      \velide
      </instantiations>
      <outputs>
      \velide
      </outputs>
      <includes>
      \velide
      </includes>
    </filter-sci>
  </filter>
\end{alltt}

Each element is discussed below.


\subsection{Element \xmlstarttag{filter}}

Element \xmlstarttag{filter} is the top-level element.  All other
elements are enclosed in element \xmlstarttag{filter}:

\begin{alltt}
  <filter>
  \velide
  </filter>
\end{alltt}

\subsection{Element \xmlstarttag{include}}

Element \xmlstarttag{include} must be used one time to reference the
ITK filter description file.  It can be used a second time to
reference an optional GUI description file.

The path to an ITK filter description file (or a dialog description
file) is provided by element \xmlstarttag{include}'s ``href''
attribute.  The file path is relative to the  \sr{} Insight
package directory (\directory{SCIRun/src/Packages/Insight}).  Note that
\xmlstarttag{include} is an empty element, closed with
characters \verb|/>|:

\begin{alltt}
  <include href="\replaceable{file\_path}"/>
\end{alltt}

For example:

\begin{alltt}
  <include href="ITK/itk_WatershedImageFilter.xml"/>
  <include href="Dataflow/Modules/Filters/XML/gui_WatershedImageFilter.xml"/>
\end{alltt}

Two \xmlstarttag{include} element's are used above.  The first
references the ITK filter description file, the second references a
GUI description file.


\subsection{Element \xmlstarttag{filter-sci}}

Element \xmlstarttag{filter-sci} provides module specific information.

\begin{alltt}
  <filter-sci name="\replaceable{module\_name}">
    <package>\replaceable{package\_name}</package>
    <category>\replaceable{category\_name}</category>
    <instantiations>
    \velide
    </instantiations>
    <outputs>
    \velide
    </outputs>
    <includes>
    \velide
    </includes>
  </filter-sci>
\end{alltt}

The value of the required \xmlattrname{name} attribute determines the name of
the \sr{} module produced from the filter, and the name of files generated. 

For example:

\begin{alltt}
  <filter-sci name="WatershedImageFilter">
  \velide
  </filter-sci>
\end{alltt}

generates a module named \module{WatershedImageFilter}.

Elements \xmlstarttag{package} and \xmlstarttag{category} specify
the package and category to which a module belongs:

\begin{alltt}
  <package>\replaceable{package\_name}</package>
  <category>\replaceable{category\_name}</category>
\end{alltt}

For example:

\begin{alltt}
  <package>Insight</package>
  <category>Filters</category>
\end{alltt}

\subsubsection{Element \xmlstarttag{instantiations}}

Element \xmlstarttag{instantiations} declares filter instantiations
that can replace those declared in an ITK filter description file (see
\secref{Element \xmlstarttag{templated}}{sec:itk_mods:templated}):

\begin{alltt}
  <instantiations use-defaults="off">
    <instance>
      <type name="\replaceable{template\_parameter\_name1}">
        <value>data\_type\_for\_parameter\_name1</value>
      </type>
    </instance>
    \velide  
  </instantiations>
\end{alltt}

When attribute \xmlattrname{use-defaults} has the value "off" then
instantiations declared in the \xmlstarttag{instantiations} element
replace instantiations declared in the \xmlstarttag{templated} element
of the ITK filter description file.  When attribute
\xmlattrname{use-defaults} has the value "on" then instantiations
declared in the \xmlstarttag{templated} element of the ITK filter
description file replace those declared in the
\xmlstarttag{instantiations} element of the SCIRun filter description
file.

Each \xmlstarttag{instance} element declares one filter instantiation.
Each \xmlstarttag{type} element corresponds to a
\xmlstarttag{template} element in an ITK filter description.  The
value of \xmlstarttag{type}'s \xmlattrname{name} attribute must match
the content of a corresponding \xmlstarttag{template} element.
Element \xmlstarttag{value} declares a data type associated with
a template parameter.  

For example:

\begin{alltt}
  <instance>
    <type name="InputImageType">
      <value>itk::Image&lt;float, 2&gt;</value>
    </type>
    <type name="OutputImageType">
      <value>itk::Image&lt;float, 2&gt;</value>
    </type>
   </instance>
   <instance>
    <type name="InputImageType">
      <value>itk::Image&lt;float, 3&gt;</value>
    </type>
    <type name="OutputImageType">
      <value>itk::Image&lt;float, 3&gt;</value>
    </type>
  </instance>
\end{alltt}


\subsubsection{Element \xmlstarttag{outputs}}

Element \xmlstarttag{outputs} is optional.  \xmlstarttag{Outputs}
lists outputs that are allowed to send \dfn{intermediate} data.

Usually data are sent on a port only after a module has finished its
execution.  When sending intermediate data is allowed, however, a
module can send data every $n_{th}$ iteration of a calculation.

Intermediate data are useful when using, for example, any of the
level-set segmentation filters---the user can visualize the
segmentation region growing or shrinking in real-time.

The \xmlstarttag{outputs} element contains one or more
\xmlstarttag{output} elements (but no more than are in the
\xmlstarttag{outputs} element of the ITK filter description).  Each
\xmlstarttag{output} element names an output and determines if the
output is allowed to send intermediate data:

\begin{alltt}
  <outputs>
    <output name="\replaceable{output\_name}"
            send_intermediate="\replaceable{yes_or_no}"/>
  </outputs>
\end{alltt}

For example:

\begin{alltt}
  <outputs>
    <output name="OutputImage" send_intermediate="yes"/>
  </outputs>
\end{alltt}

Each \xmlstarttag{output} element corresponds to an ITK filter
description \xmlstarttag{output} element.  The values of the
respective \xmlattrname{name} attributes must match.  The value of an
\xmlstarttag{output} element's \xmlattrname{send\_intermediate}
attribute is set to \xmlattrvalue{yes} if the output is allowed to
send intermediate data and \xmlattrvalue{no} otherwise.

Note that users determine, via a module's GUI, if an output that is
\emph{allowed} to send intermediate data, \emph{does} send
intermediate data.  For each output that is allowed to send
intermediate data, a module's default GUI will contain: a checkbox
that determines if the output will generate intermediate data; and a
text entry widget specifying the number of iterations between
sends of intermediate data.

For an example see file:

\begin{alltt}
  src/Packages/Insight/Dataflow/Modules/Filters/XML/
    sci\_ThresholdSegmentationLevelSetImageFilter.xml
\end{alltt}

\subsubsection{Element \xmlstarttag{includes}}

Element \xmlstarttag{includes} provides a list of include files needed
by a module's implementation.  Each \xmlstarttag{file} element
provides the path name of one include file:

\begin{alltt}
  <includes>
    <file>\replaceable{include\_file\_path\_name}</file>
    \velide
  </includes>
\end{alltt}

For example:

\begin{alltt}
  <includes>
    <file>Packages/Insight/Dataflow/Ports/ITKDatatypePort.h</file>
  </includes>
\end{alltt}

Include file paths are relative to the \sr{} \directory{src} directory
(\directory{SCIRun/src}).

\section{\sr{} GUI Description}
\label{sec:itk_mods:sr_gui_desc}

A module GUI allows access to ITK filter parameters.  By default, a
GUI is created that associates a text entry widget with each filter
parameter.  By writing a GUI description file, the user can replace
all or part of the default GUI.  

\sr{} GUI description files are located in directory:

\begin{alltt}
  SCIRun/src/Packages/Insight/Dataflow/Modules/Filters/XML
\end{alltt}

\sr{} GUI filter description files follow the naming
convention:

\begin{alltt}
  gui\_\replaceable{filter_name}.xml
\end{alltt}

for example:

\begin{alltt}
  gui\_BinaryThresholdImageFilter.xml
\end{alltt}

The structure of the GUI description file consists of the
top-level element \xmlstarttag{filter-gui} that encloses one or more
\xmlstarttag{param} elements.  Each \xmlstarttag{param} element
associates a filter parameter with a GUI widget:

\begin{alltt}
  <?xml version="1.0"  encoding="iso-8859-1"?>
  <!DOCTYPE filter-gui SYSTEM "gui_filter.dtd">
  <filter-gui name="\replaceable{name}">
    <param name="\replaceable{parameter\_name}">
      <\replaceable{widget\_element}>
      \velide
      </\replaceable{widget\_element}>
      \velide
    </param>
    \velide  
  </filter-gui>
\end{alltt}

The value of a \xmlstarttag{param} element's \xmlattrname{name}
attribute must match the value of the corresponding
\xmlstarttag{name} element in the filter description file (see
\secref{Element \xmlstarttag{parameters}}{sec:itk_mods:param_element})

One or more widgets can be specified in any order. Widgets are
displayed top to bottom in the GUI window in the order specified in
the ITK filter description file. Note that a text entry widget is
generated for each parameter lacking a \xmlstarttag{param} element.

\replaceable{Widget\_element} is one of text-entry, checkbutton,
scrollbar, radiobutton, or const. Notice that each
\replaceable{widget\_element}, except \xmlstarttag{const}, must
contain a \xmlstarttag{default} element.

A text entry widget allows the user to enter an arbitrary string of text,
and is given an initial (default) value:

\begin{alltt}
  <text-entry>
    <default>\replaceable{default\_string}</default>
  </text-entry>
\end{alltt}

A check button widget allows the user to select between true and false
values.  A check button widget is given a default value of 0 (false)
or 1 (true):

\begin{alltt}
  <checkbutton>
    <default>\replaceable{0\_or\_1}</default>
  </checkbutton>
\end{alltt}

A scrollbar widget allows the user to select one value from a
range of values.  A scrollbar widget is given a min value, a max
value, an optional scroll step value, and an initial value:

\begin{alltt}
  <scrollbar>
    <min>\replaceable{min\_value}</min>
    <max>\replaceable{max\_value}</max>
    <step>\replaceable{step\_value}</step>
    <default>\replaceable{default\_value}</default>
  </scrollbar>
\end{alltt}

If the \xmlstarttag{step} element is not present, the scroll step value
defaults to 1.

A radiobutton is a set of button widgets that allow the user to select
one value from a set of values.  Each value in a set is represented
by a button widget.  Each button has a label and a value.  A
\xmlstarttag{default} element provides \xmlstarttag{radiobutton}'s
initial value.  The initial value must be a value specified in a
\xmlstarttag{button} element:

\begin{alltt}
  <radiobutton>
    <button>
      <label>\replaceable{label\_string}</label>
      <value>\replaceable{value}</value>
    </button>
    \velide  
    <default>\replaceable{initial\_value}</default>
  </radiobutton>
\end{alltt}

A ITK filter parameter may be set to a constant value using element
\xmlstarttag{const}.  No GUI widget will be generated for a constant
parameter and the user cannot change the value of the parameter.

\begin{alltt}
  <const value="\replaceable{initial\_value}"</const>
\end{alltt}

\section{Putting It All Together}
\label{sec:itk_mods:put_together}

To build \sr{} modules from ITK filters:

\begin{enumerate}
\item Ensure that \sr{} is Configured with support for the Insight
  package.  See the \sr{} Installation Guide for details.
\item Place ITK filter description files in:
  \begin{alltt}
    SCIRun/src/Packages/Insight/ITK
  \end{alltt}
\item Place \sr{} filter description files in:
  \begin{alltt}
    SCIRun/src/Packages/Insight/Dataflow/Modules/Filters/XML
  \end{alltt}
\item Place \sr{} GUI description files in:
  \begin{alltt}
    SCIRun/src/Packages/Insight/Dataflow/Modules/Filters/XML
  \end{alltt}
\item Invoke \command{gnumake} to build \sr{} (see the \sr{}
  Installation Guide for details)
\end{enumerate}

After building \sr{}, ITK filter modules can be created by selecting
their names from the \menu{Filters} sub-menu of \sr{}'s \menu{Insight}
package menu.

\section{Examples}
\label{sec:itk_mods:example}

Examples of ITK, \sr{}, and GUI filter description files in increasing
order of complexity are:

\begin{itemize}
\item DiscreteGuassianImageFilter: Typical inputs and outputs and only
  3 parameters.  See files:
  \begin{alltt}
    \filename{itk\_DiscreteGuassianImageFilter.xml}
    \filename{sci\_DiscreteGuassianImageFilter.xml}
    \filename{gui\_DiscreteGuassianImageFilter.xml}
  \end{alltt}
\item ThresholdSegmentationLevelSetImageFilter: More parameters and a
  more complex GUI.  See files:
  \begin{alltt}
    \filename{itk\_ThresholdSegmentationLevelSetImageFilter.xml}
    \filename{sci\_ThresholdSegmentationLevelSetImageFilter.xml}
    \filename{gui\_ThresholdSegmentationLevelSetImageFilter.xml}
  \end{alltt}
\item MeanImageFilter: GUI dependent on input image dimension.  See
  files:
  \begin{alltt}
    \filename{itk\_MeanImageFilter.xml}
    \filename{sci\_MeanImageFilter.xml}
    \filename{gui\_MeanImageFilter.xml}
  \end{alltt}
\end{itemize}

Example networks using the above modules are located in
\directory{SCIRun/src/Packages/Insight/nets}.

