% -*-latex-*-
%
%  The contents of this file are subject to the University of Utah Public
%  License (the "License"); you may not use this file except in compliance
%  with the License.
%
%  Software distributed under the License is distributed on an "AS IS"
%  basis, WITHOUT WARRANTY OF ANY KIND, either express or implied. See the
%  License for the specific language governing rights and limitations under
%  the License.
%
%  The Original Source Code is SCIRun, released March 12, 2001.
%
%  The Original Source Code was developed by the University of Utah.
%  Portions created by UNIVERSITY are Copyright (C) 2001, 1994
%  University of Utah. All Rights Reserved.
%

%%%%%%%%%%  Figures used in this file %%%%%%%%%%%%%%%%%%%%%%%%%%%%%%%%
%% The basic viewer window
%begin{latexonly}
  \newcommand{\viewerwindow}%
  {\centerline{\epsfig{file=Figures/viewwindow.eps.gz,height=4in,
  bbllx=0, bblly=0, bburx=660, bbury=660}}}
%end{latexonly}
\begin{htmlonly}
  \newcommand{\viewerwindow}{%
  \htmladdimg[align=top,width=645,alt="module"]
  {../Figures/viewwindow.gif}}
\end{htmlonly}

%% View of the extended viewer window
%begin{latexonly}
  \newcommand{\extendedwindow}%
  {\centerline{\epsfig{file=Figures/viewwindow-ext.eps.gz,height=4in,
  bbllx=16, bblly=310, bburx=600, bbury=703}}}
%end{latexonly}
\begin{htmlonly}
  \newcommand{\extendedwindow}{%
  \htmladdimg[align=top,width=649,alt="extended view window"]
  {../Figures/viewwindow-ext.jpg}}
\end{htmlonly}

%% Gauge widget image
%begin{latexonly}
  \newcommand{\gaugewidget}%
  {\centerline{\epsfig{file=Figures/widget-gauge.eps.gz,height=2in,
  bbllx=0, bblly=0, bburx=457, bbury=340}}}
%end{latexonly}
\begin{htmlonly}
  \newcommand{\gaugewidget}{%
  \htmladdimg[align=top,width=458,alt="gaugewidget"]
  {../Figures/widget-gauge.gif}}
\end{htmlonly}

%% Frame widget image
%begin{latexonly}
  \newcommand{\framewidget}%
  {\centerline{\epsfig{file=Figures/widget-frame.eps.gz,height=2in,
  bbllx=0, bblly=0, bburx=328, bbury=268}}}
%end{latexonly}
\begin{htmlonly}
  \newcommand{\framewidget}{%
  \htmladdimg[align=top,width=329,alt="framewidget"]
  {../Figures/widget-frame.gif}}
\end{htmlonly}

%% Box widget image
%begin{latexonly}
  \newcommand{\boxwidget}%
  {\centerline{\epsfig{file=Figures/widget-box.eps.gz,height=2in,
  bbllx=0, bblly=0, bburx=458, bbury=342}}}
%end{latexonly}
\begin{htmlonly}
  \newcommand{\boxwidget}{%
  \htmladdimg[align=top,width=459,alt="boxwidget"]
  {../Figures/widget-box.gif}}
\end{htmlonly}

%% Ring widget image
%begin{latexonly}
  \newcommand{\ringwidget}%
  {\centerline{\epsfig{file=Figures/widget-ring.eps.gz,height=2in,
  bbllx=0, bblly=0, bburx=507, bbury=467}}}
%end{latexonly}
\begin{htmlonly}
  \newcommand{\ringwidget}{%
  \htmladdimg[align=top,width=508,alt="ringwidget"]
  {../Figures/widget-ring.gif}}
\end{htmlonly}

%%%%%%%%%%%%%%%%%%%%%%%%%%%%%%%%%%%%%%%%%%%%%%%%%%%%%%%%%%%%%%%%%%%%%%
\newcommand{\graphics}{\emph{Graphics}}

\section{Visualization with the \viewer{}}
\label{sec:viewer}
\index{Viewer@\viewer{}}

This section describes the most frequently used \sr{} module,
the \viewer{}, which has the task of displaying interactive graphical
output to the computer screen.  You will use the \viewer{} any time you
wish to see a geometry, or spatial data.  More important for
computational steering (described in \secref{Computational
Steering}{sec:con-steering}) is that the \viewer{} provides access to
many simulation parameters and controls and thus indirectly initiates new
iterations of the simulation steps.

We begin with an overview of the \viewer{} window and its controls, then
describe in detail all the options and variations.

\subsection{Anatomy of the \viewer{} Window}
\label{sec:viewer-anatomy} 
\index{Viewer@\viewer{}!anatomy}

The \viewer{} window contains two main areas, the upper portion,
called the \graphics{} window, which displays the graphics, and the
lower portion, where most of the control buttons are.
Figure~\ref{fig:viewwindow} contains an example of a \viewer{} window.
In the \graphics{} window, viewing is controlled mostly by means of the
mouse, mouse buttons, and various modifier keys (shift/control/alt).
In the lower window are a lot of buttons and sliders, the function of
which will become clear as you read this section of the manual.

\begin{figure}[htb]
  \begin{makeimage}
  \end{makeimage}
  \viewerwindow
%    \framebox{\parbox[3in]{\columnwidth}{The\dotfill Figure\\
%    \vspace{2in}\\
%    With some \dotfill dummy text}}
  \caption{\label{fig:viewwindow} The default \viewer{} window in \SR{}}
\end{figure}


First, try out the controls for the \graphics{} window by moving the mouse
to the center of the viewer window and clicking and holding the left button
and then dragging the mouse.  The objects should translate along with the
mouse.  Do the same operation with the middle mouse button and the objects
will rotate around a point in the center of the display.  The right mouse
button controls the scale of the display, zooming in when the mouse moves
downward or to the right.  See \secref{Mouse Control in the
Viewer}{sec:view-mouse} for details on using the mouse.

The controls visible along the bottom of the \viewer{} window set some basic
configurations as follows:
%
\begin{description}
  \buttondesc{Autoview} Restores the display to a default
  condition. Very useful when some combination of settings results in
  the objects disappearing from the view window.
  
  \buttondesc{Set Home View} Captures the setting of the
  current view so you can return to it later by clicking the Go home''
  button.

  \buttondesc{Go home} Restore the current home view.
  
  \buttondesc{Views} Lists a number of standard viewing angles
  and orientations.  The view directions align with the Cartesian axes
  of the objects and the ``Up vector'' choice sets the orientation of
  the objects when viewed along the selected axis.
\end{description}

In the left corner of the control panel of the \viewer{} window are
performance indicators that document the current rendering speed for the
display.  The better the graphics performance of the workstation you
have, the higher the display rate.

You may reveal more controls by clicking the
\latexhtml{\fbox{+}}{\button{[+]}} button in the lower right corner of
the \viewer{} window.  The extended controls are described in
\secref{Extended control window} {sec:view-control}.


\subsubsection{Menus}
\index{Viewer@\viewer{}!menus}

At the top of the \viewer{} window are three pull-down menus.

\begin{description}
  \menuitemdesc{File} Current contains only the \menuitem{Save
    image file...} item which allows you to save the contents of the
  \viewer{} window as an image to a file.
  
  \menuitemdesc{Edit} Provides access to controls for the
  background color for the window, as well as the clipping planes
  (requires the ``Use Clip'' control to be selected in the extended
  controls described in \secref{Extended control
    window}{sec:view-control}).
  
  \menuitemdesc{Visual} Allows you to select between different
  graphics hardware settings that are available on your workstation.
  The list is ordered heuristically from most to least useful.
\end{description}

\subsection{Mouse Control in the \viewer{} Window}
\label{sec:view-mouse} 
\index{Viewer@\viewer{}!mouse controls}

The mouse controls within \SR{} are extensive and flexible.  The
resulting action depends on the choice of mouse button, any
simultaneously pressed control keys, and the way the mouse moves.  The
description in Tables~\ref{tab:view-mouse} and~\ref{tab:view-unicam}
below may seem overly complicated at first, but with a little playing,
it becomes intuitive (another way of saying you will learn it if you
use it enough).

\begin{table}[htb]
\begin{center}
  \begin{tabular}{|l|l|p{5in}|} \hline
    \multicolumn{3}{|c|}{\large Mouse Controls}\\ \hline \hline 
    \multicolumn{1}{|c|}{Control Key} & 
    \multicolumn{1}{|c|}{Button} & 
    \multicolumn{1}{|c|}{Action}\\ \hline
None & Left & Translate scene \\
     & Middle & \begin{raggedleft} Rotate scene about its center on an arc
    ball that surrounds it; rotation direction is a function of the
    initial mouse location so try different sites and note the
    response. \end{raggedleft}\\  
     & Right & Zoom or scale scene (downwards and to the right increases
     size, upwards or to the left decreases size) \\ \hline
Shift & Left & Select and move a widget in the display \\
      & Middle & Toggle through the modes for a widget \\
      & Right & Pop up a widget information window \\ \hline
Control & Left & Translate in the Z-direction, \ie{} zoom in and out of the
    screen (down moves closer, up further away).  Moving left and
    right increases the ``throttle'' of the Z-direction motion.  If
    the cursor is over a point on an object when clicked, this point
    becomes the center of the screen for translation.\\ 
      & Middle & Rotate the camera view about the eye point (using arcball
    motion). \\ 
      & Right & Unicam movement (see Table~\ref{tab:view-unicam})\\ \hline
\end{tabular}
\caption{\label{tab:view-mouse} Mouse controls for the \viewer{}}
\end{center}
\end{table}

\bigskip

\begin{table}
\begin{center}
\begin{tabular}{|l|l|p{3in}|} \hline
    \multicolumn{3}{|c|}{\large Unicam movement (Control key and right mouse
    button} \\ \hline \hline
    \multicolumn{1}{|c|}{Initial mouse location} & 
    \multicolumn{1}{|c|}{Action} & \\ 
    \hline
    Near edge of display & Rotate objects on the arc ball & \\
    Near the objects & Following behavior: & \\
    \hline
    & \multicolumn{1}{|c|}{Initial mouse movement} & 
    \multicolumn{1}{|c|}{Action}\\ \hline
    & Horizontal & Pan objects \\ 
    & Vertical & Zoom and pan: down = zoom in, up = zoom
    out, left and right= pan left and right) \\
    & None & Set rotation point for subsequent arc ball rotation.\\
    \hline
\end{tabular}
\caption{\label{tab:view-unicam} Autocam mouse controls in the \viewer{}}
\end{center}
\end{table}



\subsection{Extended Control Window}
\label{sec:view-control} 
\index{Viewer@\viewer{}!extended controls}

Click on the \button{[+]} button in the lower right corner of the
default \viewer{} window, and the window expands to reveal an extended
panel of control buttons, as shown in Figure~\ref{fig:extviewwindow}.
Click on the \button{[-]} sign that now replaces the \button{[+]} and
this extended panel disappears again.  Here we describe the control
options available in the extended control window.

\begin{figure}[htb]
  \begin{makeimage}
  \end{makeimage}
  \extendedwindow
%    \framebox{\parbox[3in]{\columnwidth}{The\dotfill Figure\\
%    \vspace{2in}\\
%    With some \dotfill dummy text}}
  \caption{\label{fig:extviewwindow} The lower portion of extended
    \viewer{} window in \SR{}} 
\end{figure}


\subsubsection{Object Selector}

The lower portion of the extended \viewer{} window is divided into three
columns. The middle column contains a list of all the objects in the
display.  If the list becomes long enough, a scroll bar on the left
hand side controls which are visible.  For each entry in the list, we have
the following controls, reading from left to right:

\begin{itemize}
  \item At the left end of each of the
        objects in the list is a selection indicator that displays red
        when that object is 
        selected.  The \viewer{} window only displays those objects that
        are selected.
  \item Next comes the name of the object.
  \item The \button{Shading} control box that comes next determines the
        shading options that will be used for rendering the object.
        Options include: Lighting, BBox, Fog, Use Clip, Back Cull, and
        Display List (for descriptions, see
        \secref{Rendering controls}{sec:view-rendering}).
  \item At the right end of each entry is the lighting control, initially
        marked \menu{Default}.  In the Default setting, the common
        rendering controls described in \secref{Rendering
        controls}{sec:view-rendering} below apply.  Clicking this box
        reveals a set of options that will apply only to this object that
        include Wire, Flat and Gouraud.
\end{itemize}


\subsubsection{Rendering Controls}
\label{sec:view-rendering} 
\index{Viewer@\viewer{}!rendering}

The left column of the extended \viewer{} window contains controls that
apply to all of the selected objects with ``Default'' lighting selected.
Those without the Default setting will use their own, object specific
settings, as described in the previous section.  The lighting and shading
options available are:
%
\begin{description}
  \descitem{Lighting} Toggles whether or not the \viewer{} applies
  lighting to the display.  Objects without lighting have a constant
  color.
        
  \descitem{Fog} Draws objects with variable intensity based on their
  distance from the user, also known as ``depth cueing''.  Close
  objects appear brighter while more remote objects fade gradually
  into the background as a function of distance from the front.
  
  \descitem{BBox} Toggles whether the \viewer{} draws the selected
  objects in full detail or as a simple bounding box.
  
  \descitem{Use Clip} Applies up to six clipping planes to the
  display.  To control the clipping plane locations, use the ``Edit
  -\ra{} Clipping Planes'' menu at the top of the \viewer{} window.
  
  \descitem{Back Cull} Displays only the forward facing facets of any
  surface objects in the display.
  
  \descitem{Display List} Cache the list of objects to be displayed;
  this option accelerates rendering when the content of the display
  does not change.

  \descitem{Shading} Selects the type of shading for objects from the
        following options:
        \begin{description}
          \descitem{Wire} Show only the wire mesh of objects.
          \descitem{Flat} Draw each facet with a constant color.
          \descitem{Gouraud} Linearly interpolate the color across facets. 
        \end{description}
\end{description}

The right hand column of the extended \viewer{} window contains controls
for displaying the axes and creating stereoscopic rendering.  

\paragraph{Stereo viewing: } requires hardware LCD glasses synchronized
with the display so that visibility for each eye coincides with the
display of the appropriate view.  The ``Fusion Scale'' control provides a
means of setting the eye separation and thus setting the view that is most
suited to facial anatomy and distance from the screen.

\subsubsection{Making Movies}
\label{sec:view-movies} 
\index{movies}

The \viewer{} window in \SR{} has simple controls for capturing sequences
of images into animations or movies.  Here we describe how this works.

In the left column of the extended \viewer{} window are controls for
selecting movie type and then initiating and stopping the acquisition of
individual frames in the movie.

\SR{} sends a frame to the movie after each ``redraw'' operation, \ie{}
each time anything moves in the display or any visualization parameter
changes.  If the MPEG package is available (See the
\htmladdnormallink{Installation Manual}{\installguideurl} for
details) then an option will be available for saving the animations as MPEG
movies.

There is also a button that forces the size of the graphical window to be
352x240 pixels in size, which is a standard format well suited to MPEG.

\subsection{Control Widgets}
\label{sec:view-widgets} 

While the \hyperref{mouse controls}{mouse controls in
Section}{}{sec:view-mouse} describe many ways to interact with the contents
of the \viewer{}, SR{} also supports some powerful display widgets.
Examples of widgets capabilities include managing cutting surfaces colored
according to the local data values, displaying streamlines in vector
fields, or selecting sub-volumes within the display area for further
manipulation. 
 
We have tried to make interacting with these widgets as consistent as
possible so that, for example, controlling parameters is usually by clicking
and dragging on either a cylindrical ``collar'' or a sphere element of the
widget.  
%%The original design of these modules was by James Purcifal
%%\cite{??}. 
Note that a single widget may have more than one purpose
depending on the context in which it exists.

In this section, we describe the widgets available within \SR{} and \BIOPSE{}.
The same widget may, for example, select a clipping or a display plane
through a three-dimensional object but may also set the seed points for a
streamline module.  

\subsubsection{Gauge Widget}
\label{sec:view-gaugewidget} 

\begin{figure}[htb]
  \begin{makeimage}
  \end{makeimage}
  \gaugewidget
%    \framebox{\parbox[3in]{\columnwidth}{The\dotfill Figure\\
%    \vspace{2in}\\
%    With some \dotfill dummy text}}
  \caption{\label{fig:gaugewidget} The gauge widget for setting location and
    density of seed points}
\end{figure}

\paragraph{Appearance} The Gauge
Widget (see Figure~\ref{fig:gaugewidget}) consists of two spheres (A) connected by a cylinder (B) with a
small slider collar (C) on the cylinder.  There are also small resize
cylinders extending from the spheres (D).

\paragraph{Purpose} The primary use of the Gauge Widget is to set the
location and density of streamlines emerging from the long cylinder.  It
may also be used as a more general purpose three-dimensional slider or as a
source for a stream surface. 

\paragraph{Controls} Clicking and dragging either sphere causes the
entire widget to move in space, rotating about the other sphere and
following along behind the selected sphere.  Dragging either of the resize
cylinders cases the size of the widget to change and dragging any point on
the main cylinder moves the whole widget without any change in orientation.
Dragging the slider collar changes the associated value, typically the
density of seed points for a streamline source.

\subsubsection{Frame Widget}
\label{sec:view-framewidget} 

\begin{figure}[htb]
  \begin{makeimage}
  \end{makeimage}
  \framewidget
%    \framebox{\parbox[3in]{\columnwidth}{The\dotfill Figure\\
%    \vspace{2in}\\
%    With some \dotfill dummy text}}
  \caption{\label{fig:framewidget} The frame widget for selecting
    cutting/projection planes}
\end{figure}


\paragraph{Appearance} The
Frame Widget(See Figure~\ref{fig:framewidget}) consists of four cylinders connected in a rectangle.  In
the middle of each of the cylinders there is a sphere (B), from which
a resize cylinder extends (C).

\paragraph{Purpose} The primary uses of the Frame Widget is for image
plane definition, for defining stream volumes, and as a ``tie dye'' as with
the Ring Widget described in \secref{Ring Widget}{sec:view-ringwidget}.

\paragraph{Controls} Clicking and dragging a sphere on the widget will
cause the widget to rotate about its center; dragging on a resize cylinder
will move the associated edge and thus extend or contract the rectangle.
Dragging any of the cylinders will drag the entire widget through space.


\subsubsection{Box Widget}
\label{sec:view-boxwidget} 

\begin{figure}[htb]
  \begin{makeimage}
  \end{makeimage}
  \boxwidget
%    \framebox{\parbox[3in]{\columnwidth}{The\dotfill Figure\\
%    \vspace{2in}\\
%    With some \dotfill dummy text}}
  \caption{\label{fig:boxwidget} The boxwidget for selecting sub-volumes}
\end{figure}

\paragraph{Appearance} The Box
Widget (see Figure~\ref{sec:view-boxwidget}) consists of twelve 
connected cylinders (A) which form a hexahedral box (three-dimensional
rectangle) with cylinders indicating the edges of the box.  In the
middle of each face is a sphere (B) with an attached short cylinder
(C) providing resize control.

\paragraph{Purpose} The primary use of the Box Widget is to select a
subvolume of the workspace for further manipulation (\eg{} volume
rendering, isosurfaces, streamlines, mesh adaption) where the faces of the
widget act as orthogonal clipping planes.

\paragraph{Controls} Clicking and dragging on one of the spheres rotates
the widget about its center without changing the position of the center.
Clicking on and dragging any resize handle
% What do these look like??
causes the associated face to extend without changing its orientation.
Dragging a cylinder causes the entire widget to move without changing its
orientation.

\subsubsection{Ring Widget}
\label{sec:view-ringwidget} 

\begin{figure}[htb]
  \begin{makeimage}
  \end{makeimage}
  \ringwidget
%    \framebox{\parbox[3in]{\columnwidth}{The\dotfill Figure\\
%    \vspace{2in}\\
%    With some \dotfill dummy text}}
  \caption{\label{fig:ringwidget} The ring widget for selecting
    cutting/projection planes}
\end{figure}


\paragraph{Appearance} The Ring
Widget (see Figure~\ref{sec:view-ringwidget}) consists of a ring (A)
with four embedded spheres (B), each with a resize cylindrical
attached (D).  Between two of the spheres is a sliding collar (C).
One of the resize cylinders has a special material property (typically
a different color from the other cylinders) to indicate that it is the
``halfway point'' for the slider (E).

\paragraph{Purpose} The primary use of the Ring Widget is to set the
density of streamlines emerging from the ring---the ring serves as a set of
seed points from which will emerge streamlines.  The Ring Widget can also
serve as a three-dimensional angle gauge, as a source for multiple
streamlines throughout its surface, as a source for a stream surface from
the outer ring, and as a source for a stream volume.  Another use is as a
color sheet, or ``tie dye'', in which the surface is colored as a function of
the scalar value of the field at each point.

\paragraph{Controls} Clicking and dragging the slider collar along the
ring changes the density of the seed points or some other related
parameter.  Dragging the spheres controls the orientation of the Ring
Widget, while moving the resize cylinders changes the radius of the Ring
Widget about its center.  Dragging any other point on the ring moves the
ring in space without changing its radius or orientation.


%%% Local Variables: 
%%% mode: latex
%%% TeX-master: t
%%% End: 
