\documentclass[twoside,10pt,a4paper]{article}
%\usepackage[T1]{fontenc}
%\usepackage[urw-garamond]{mathdesign}
\usepackage{helvet}
\usepackage[pdftex]{graphicx}
\usepackage[usenames]{color}
\usepackage{cancel}
%\usepackage{mathtools}
%\newcommand\hmmax{-1} % default 3
%\newcommand\bmmax{4} % default 4
\usepackage{bm}
\usepackage{amsmath}
%\usepackage{amssymb}
\usepackage{amstext}
%\usepackage{amsfonts}
\usepackage{lastpage}
\usepackage[sf,bf,compact,topmarks,calcwidth,pagestyles]{titlesec}
\usepackage{titletoc}
\usepackage{fancyhdr}
\usepackage{datetime}
\usepackage[margin={1.5cm,1.5cm},font=normalsize,format=hang,labelfont=bf,textfont=it,labelsep=endash]{caption}
\usepackage{setspace}
\usepackage{boxedminipage}
%\usepackage{stmaryrd}
\usepackage{pxfonts}
\usepackage[normalem]{ulem}
\usepackage{algorithm}
\usepackage{algorithmic}

% The page layout plan.
\setlength{\parskip}       {8pt}
\setlength{\oddsidemargin} {-10mm}
\setlength{\evensidemargin}{-10mm}
\setlength{\topmargin}     {-15mm}
\setlength{\headheight}    {10mm}
\setlength{\headsep}       {5mm}
\setlength{\textheight}    {250mm}
\setlength{\textwidth}     {180mm}
\setlength{\topskip}       {12pt}
\setlength{\footskip}      {10mm}
\setlength{\marginparsep}  {0mm}
\setlength{\marginparwidth}{0mm}
\setlength{\marginparpush} {0mm}
\raggedright

\definecolor{red}{rgb}{1,0,0}
\definecolor{green}{rgb}{0,1,0}
\definecolor{blue}{rgb}{0,0,1}
\definecolor{Red}{rgb}{0.8666,0.03137,0.02352}
\definecolor{Blue}{rgb}{0.00784,0.67059,0.91764}
\definecolor{Darkgreen}{rgb}{0,0.68235,0}
\definecolor{Green}{rgb}{0,0.8,0}
\definecolor{Bl}{rgb}{0,0.2,0.91764}
\definecolor{Royalblue}{rgb}{0,0.2,0.91764}
\definecolor{Brickred}{rgb}{0.644541,0.164065,0.164065}
\definecolor{Brown}{rgb}{0.6,0.4,0.4}
\definecolor{Orange}{rgb}{1,0.647059,0}
\definecolor{Indigo}{rgb}{0.746105,0,0.996109}
\definecolor{Violet}{rgb}{0.308598,0.183597,0.308598}
\definecolor{Lightgrey}{rgb}{0.762951,0.762951,0.762951}
\definecolor{Darkgrey}{rgb}{0.503548,0.503548,0.503548}
\definecolor{Pink}{rgb}{1,0.6,0.6}
\definecolor{MyLightMagenta}{cmyk}{0.1,0.8,0,0.1}
\definecolor{MyDarkBlue}{rgb}{0,0.08,0.45}

\pagestyle{myheadings}

\newcommand{\Red}{\color{Brickred}}
\newcommand{\Blue}{\color{Royalblue}}
\newcommand{\Green}{\color{Darkgreen}}
\newcommand{\Violet}{\color{Violet}}
\newcommand{\Jump}[1]{\ensuremath{\llbracket#1\rrbracket}}
\newcommand{\Blimitx}[1]{\ensuremath{\left[#1\right]_{x_a}^{x_b}}}
\newcommand{\Deriv}[2]{\ensuremath{\cfrac{d#1}{d#2}}}
\newcommand{\MDeriv}[2]{\ensuremath{\cfrac{D#1}{D#2}}}
\newcommand{\DDeriv}[2]{\ensuremath{\cfrac{d^2#1}{d#2^2}}}
\newcommand{\DDDeriv}[2]{\ensuremath{\cfrac{d^3#1}{d#2^3}}}
\newcommand{\DDDDeriv}[2]{\ensuremath{\cfrac{d^4#1}{d#2^4}}}
\newcommand{\Intx}{\ensuremath{\int_{x_a}^{x_b}}}
\newcommand{\IntX}{\ensuremath{\int_{X_a}^{X_b}}}
\newcommand{\Intiso}{\ensuremath{\int_{-1}^{1}}}
\newcommand{\IntOmegaA}{\ensuremath{\int_{\Omega_0}}}
\newcommand{\IntOmega}{\ensuremath{\int_{\Omega}}}
\newcommand{\IntDOmega}{\ensuremath{\int_{\partial\Omega}}}
\newcommand{\IntOmegap}{\ensuremath{\int_{\Omega'}}}
\newcommand{\Norm}[2]{\ensuremath{\left\lVert#1\right\rVert_{#2}}}
\newcommand{\Var}[1]{\ensuremath{\delta #1}}
\newcommand{\DelT}{\ensuremath{\Delta t}}
\newcommand{\CalA}{\ensuremath{\mathcal{A}}}
\newcommand{\CalB}{\ensuremath{\mathcal{B}}}
\newcommand{\CalD}{\ensuremath{\mathcal{D}}}
\newcommand{\BCalD}{\ensuremath{\boldsymbol{\CalD}}}
\newcommand{\CalF}{\ensuremath{\mathcal{F}}}
\newcommand{\CalL}{\ensuremath{\mathcal{L}}}
\newcommand{\CalM}{\ensuremath{\mathcal{M}}}
\newcommand{\BCalM}{\ensuremath{\boldsymbol{\CalM}}}
\newcommand{\CalN}{\ensuremath{\mathcal{N}}}
\newcommand{\CalP}{\ensuremath{\mathcal{P}}}
\newcommand{\CalS}{\ensuremath{\mathcal{S}}}
\newcommand{\BCalS}{\ensuremath{\boldsymbol{\CalS}}}
\newcommand{\CalT}{\ensuremath{\mathcal{T}}}
\newcommand{\CalV}{\ensuremath{\mathcal{V}}}
\newcommand{\CalW}{\ensuremath{\mathcal{W}}}
\newcommand{\CalX}{\ensuremath{\mathcal{X}}}
\newcommand{\Comp}[2]{\ensuremath{#1 \circ #2}}
\newcommand{\Map}[3]{\ensuremath{#1 : #2 \rightarrow #3}}
\newcommand{\MapTo}[3]{\ensuremath{#1 : #2 \mapsto #3}}
\newcommand{\Real}[1]{\ensuremath{\mathbb{R}^{#1}}}
\newcommand{\Ve}{\ensuremath{\varepsilon}}
\newcommand{\BHat}[1]{\ensuremath{\widehat{\boldsymbol{#1}}}}
\newcommand{\BTx}{\ensuremath{\tilde{\boldsymbol{x}}}}
\newcommand{\Beh}{\ensuremath{\hat{\boldsymbol{e}}}}
\newcommand{\BHex}{\ensuremath{\hat{\boldsymbol{e}}_1}}
\newcommand{\BHey}{\ensuremath{\hat{\boldsymbol{e}}_2}}
\newcommand{\BHez}{\ensuremath{\hat{\boldsymbol{e}}_3}}
\newcommand{\BHn}[1]{\ensuremath{\hat{\boldsymbol{n}}_{#1}}}
\newcommand{\BHe}[1]{\ensuremath{\hat{\boldsymbol{e}}_{#1}}}
\newcommand{\BHg}[1]{\ensuremath{\hat{\boldsymbol{g}}_{#1}}}
\newcommand{\BHG}[1]{\ensuremath{\hat{\boldsymbol{G}}_{#1}}}
\newcommand{\Hn}{\ensuremath{\hat{\boldsymbol{n}}}}
\newcommand{\Mba}{\ensuremath{\mathbf{a}}}
\newcommand{\Mbatilde}{\ensuremath{\widetilde{\mathbf{a}}}}
\newcommand{\Mbb}{\ensuremath{\mathbf{b}}}
\newcommand{\Mbd}{\ensuremath{\mathbf{d}}}
\newcommand{\Mbf}{\ensuremath{\mathbf{f}}}
\newcommand{\Mbn}{\ensuremath{\mathbf{n}}}
\newcommand{\Mbntilde}{\ensuremath{\widetilde{\mathbf{n}}}}
\newcommand{\Mbr}{\ensuremath{\mathbf{r}}}
\newcommand{\Mbu}{\ensuremath{\mathbf{u}}}
\newcommand{\Mbv}{\ensuremath{\mathbf{v}}}
\newcommand{\Mbx}{\ensuremath{\mathbf{x}}}
\newcommand{\MbA}{\ensuremath{\mathbf{A}}}
\newcommand{\MbB}{\ensuremath{\mathbf{B}}}
\newcommand{\MbC}{\ensuremath{\mathbf{C}}}
\newcommand{\MbD}{\ensuremath{\mathbf{D}}}
\newcommand{\MbH}{\ensuremath{\mathbf{H}}}
\newcommand{\MbHbar}{\ensuremath{\mathbf{\overline{H}}}}
\newcommand{\MbI}{\ensuremath{\mathbf{I}}}
\newcommand{\MbK}{\ensuremath{\mathbf{K}}}
\newcommand{\MbKbar}{\ensuremath{\overline{\mathbf{K}}}}
\newcommand{\MbKtilde}{\ensuremath{\widetilde{\mathbf{K}}}}
\newcommand{\MbM}{\ensuremath{\mathbf{M}}}
\newcommand{\MbN}{\ensuremath{\mathbf{N}}}
\newcommand{\MbP}{\ensuremath{\mathbf{P}}}
\newcommand{\MbPbar}{\ensuremath{\overline{\mathbf{P}}}}
\newcommand{\MbR}{\ensuremath{\mathbf{R}}}
\newcommand{\MbT}{\ensuremath{\mathbf{T}}}
\newcommand{\MbV}{\ensuremath{\mathbf{V}}}
\newcommand{\MbX}{\ensuremath{\mathbf{X}}}
\newcommand{\MbSig}{\ensuremath{\boldsymbol{\sigma}}}
\newcommand{\Mbone}{\ensuremath{\mathbf{1}}}
\newcommand{\Mbzero}{\ensuremath{\mathbf{0}}}
%\newcommand{\Mb}{\ensuremath{\left[\mathsf{b}\right]}}
%\newcommand{\Mu}{\ensuremath{\left[\mathsf{u}\right]}}
%\newcommand{\Mv}{\ensuremath{\left[\mathsf{v}\right]}}
%\newcommand{\Mw}{\ensuremath{\left[\mathsf{w}\right]}}
%\newcommand{\Mx}{\ensuremath{\left[\mathsf{x}\right]}}
\newcommand{\MA}{\ensuremath{\left[\mathsf{A}\right]}}
\newcommand{\MB}{\ensuremath{\left[\mathsf{B}\right]}}
\newcommand{\MC}{\ensuremath{\left[\mathsf{C}\right]}}
\newcommand{\MD}{\ensuremath{\left[\mathsf{D}\right]}}
\newcommand{\MH}{\ensuremath{\left[\mathsf{H}\right]}}
\newcommand{\MI}{\ensuremath{\left[\mathsf{I}\right]}}
\newcommand{\ML}{\ensuremath{\left[\mathsf{L}\right]}}
\newcommand{\MM}{\ensuremath{\left[\mathsf{M}\right]}}
\newcommand{\MN}{\ensuremath{\left[\mathsf{N}\right]}}
\newcommand{\MP}{\ensuremath{\left[\mathsf{P}\right]}}
\newcommand{\MR}{\ensuremath{\left[\mathsf{R}\right]}}
\newcommand{\MT}{\ensuremath{\left[\mathsf{T}\right]}}
\newcommand{\MV}{\ensuremath{\left[\mathsf{V}\right]}}
%\newcommand{\Mone}{\ensuremath{\left[\mathsf{1}\right]}}
%\newcommand{\Mzero}{\ensuremath{\left[\mathsf{0}\right]}}
\newcommand{\SfA}{\ensuremath{\boldsymbol{\mathsf{A}}}}
\newcommand{\Sfc}{\ensuremath{\boldsymbol{\mathsf{c}}}}
\newcommand{\SfC}{\ensuremath{\boldsymbol{\mathsf{C}}}}
\newcommand{\SfD}{\ensuremath{\boldsymbol{\mathsf{D}}}}
\newcommand{\SfI}{\ensuremath{\boldsymbol{\mathsf{I}}}}
\newcommand{\SfL}{\ensuremath{\boldsymbol{\mathsf{L}}}}
\newcommand{\Sfp}{\ensuremath{\boldsymbol{\mathsf{p}}}}
\newcommand{\SfP}{\ensuremath{\boldsymbol{\mathsf{P}}}}
\newcommand{\SfS}{\ensuremath{\boldsymbol{\mathsf{S}}}}
\newcommand{\SfT}{\ensuremath{\boldsymbol{\mathsf{T}}}}
\newcommand{\Msig}{\ensuremath{\left[\boldsymbol{\sigma}\right]}}
\newcommand{\Meps}{\ensuremath{\left[\boldsymbol{\varepsilon}\right]}}
\newcommand{\Ex}{\ensuremath{\boldsymbol{e}_1}}
\newcommand{\Ey}{\ensuremath{\boldsymbol{e}_2}}
\newcommand{\Ez}{\ensuremath{\boldsymbol{e}_3}}
\newcommand{\Exp}{\ensuremath{\boldsymbol{e}^{'}_1}}
\newcommand{\Eyp}{\ensuremath{\boldsymbol{e}^{'}_2}}
\newcommand{\Ezp}{\ensuremath{\boldsymbol{e}^{'}_3}}
\newcommand{\Ep}{\ensuremath{\varepsilon_p}}
\newcommand{\Epi}{\ensuremath{\varepsilon_{pi}}}
\newcommand{\Epdot}[1]{\ensuremath{\dot{\varepsilon}_{p#1}}}
\newcommand{\Epsxx}{\ensuremath{\varepsilon_{11}}}
\newcommand{\Epsyy}{\ensuremath{\varepsilon_{22}}}
\newcommand{\Epszz}{\ensuremath{\varepsilon_{33}}}
\newcommand{\Epsyz}{\ensuremath{\varepsilon_{23}}}
\newcommand{\Epszx}{\ensuremath{\varepsilon_{31}}}
\newcommand{\Epsxy}{\ensuremath{\varepsilon_{12}}}
\newcommand{\Sigxx}{\ensuremath{\sigma_{11}}}
\newcommand{\Sigyy}{\ensuremath{\sigma_{22}}}
\newcommand{\Sigzz}{\ensuremath{\sigma_{33}}}
\newcommand{\Sigyz}{\ensuremath{\sigma_{23}}}
\newcommand{\Sigzx}{\ensuremath{\sigma_{31}}}
\newcommand{\Sigxy}{\ensuremath{\sigma_{12}}}
\newcommand{\Eps}[1]{\ensuremath{\varepsilon_{#1}}}
\newcommand{\Sig}[1]{\ensuremath{\sigma_{#1}}}
\newcommand{\X}{\ensuremath{X_1}}
\newcommand{\Y}{\ensuremath{X_2}}
\newcommand{\Z}{\ensuremath{X_3}}
\newcommand{\Balpha}{\ensuremath{\boldsymbol{\alpha}}}
\newcommand{\Balphahat}{\ensuremath{\widehat{\boldsymbol{\alpha}}}}
\newcommand{\Bchi}{\ensuremath{\boldsymbol{\chi}}}
\newcommand{\Beta}{\ensuremath{\boldsymbol{\eta}}}
\newcommand{\Bveps}{\ensuremath{\boldsymbol{\varepsilon}}}
\newcommand{\BGamma}{\ensuremath{\boldsymbol{\mathit{\Gamma}}}}
\newcommand{\BGammahat}{\ensuremath{\boldsymbol{\mathit{\widehat{\Gamma}}}}}
%\newcommand{\BGammahat}{\ensuremath{\widehat{\BGamma}}}
\newcommand{\Bkappa}{\ensuremath{\boldsymbol{\kappa}}}
\newcommand{\Bbeps}{\ensuremath{\bar{\boldsymbol{\varepsilon}}}}
\newcommand{\Bnabla}{\ensuremath{\boldsymbol{\nabla}}}
\newcommand{\Bomega}{\ensuremath{\boldsymbol{\omega}}}
\newcommand{\Bsig}{\ensuremath{\boldsymbol{\sigma}}}
\newcommand{\Btau}{\ensuremath{\boldsymbol{\tau}}}
\newcommand{\Bpi}{\ensuremath{\boldsymbol{\pi}}}
\newcommand{\Brho}{\ensuremath{\boldsymbol{\rho}}}
\newcommand{\Bvarphi}{\ensuremath{\boldsymbol{\varphi}}}
\newcommand{\Blambda}{\ensuremath{\boldsymbol{\lambda}}}
\newcommand{\Btheta}{\ensuremath{\boldsymbol{\theta}}}
\newcommand{\Bmu}{\ensuremath{\boldsymbol{\mu}}}
\newcommand{\Bxi}{\ensuremath{\boldsymbol{\xi}}}
\newcommand{\BPi}{\ensuremath{\boldsymbol{\Pi}}}
\newcommand{\Bone}{\ensuremath{\boldsymbol{\mathit{1}}}}
\newcommand{\Bonev}{\ensuremath{\boldsymbol{1}}}
\newcommand{\Bzero}{\ensuremath{\boldsymbol{0}}}
\newcommand{\BzeroT}{\ensuremath{\boldsymbol{\mathit{0}}}}
\newcommand{\Ba}{\ensuremath{\mathbf{a}}}
\newcommand{\Bb}{\ensuremath{\mathbf{b}}}
\newcommand{\BbT}{\ensuremath{\boldsymbol{b}}}
\newcommand{\Bc}{\ensuremath{\mathbf{c}}}
\newcommand{\Bd}{\ensuremath{\mathbf{d}}}
\newcommand{\BdT}{\ensuremath{\boldsymbol{d}}}
\newcommand{\Be}{\ensuremath{\mathbf{e}}}
\newcommand{\BeT}{\ensuremath{\boldsymbol{e}}}
\newcommand{\Bf}{\ensuremath{\mathbf{f}}}
\newcommand{\Bg}{\ensuremath{\mathbf{g}}}
\newcommand{\BTg}{\ensuremath{\boldsymbol{g}}}
\newcommand{\Bh}{\ensuremath{\mathbf{h}}}
\newcommand{\Bk}{\ensuremath{\mathbf{k}}}
\newcommand{\Bl}{\ensuremath{\mathbf{l}}}
\newcommand{\Bm}{\ensuremath{\mathbf{m}}}
\newcommand{\Bn}{\ensuremath{\mathbf{n}}}
\newcommand{\BnT}{\ensuremath{\boldsymbol{n}}}
\newcommand{\Bo}{\ensuremath{\mathbf{o}}}
\newcommand{\Bp}{\ensuremath{\mathbf{p}}}
\newcommand{\Bq}{\ensuremath{\mathbf{q}}}
\newcommand{\Br}{\ensuremath{\mathbf{r}}}
\newcommand{\Bs}{\ensuremath{\mathbf{s}}}
\newcommand{\BsT}{\ensuremath{\boldsymbol{s}}}
\newcommand{\Bt}{\ensuremath{\mathbf{t}}}
\newcommand{\Bu}{\ensuremath{\mathbf{u}}}
\newcommand{\Bv}{\ensuremath{\mathbf{v}}}
\newcommand{\Bw}{\ensuremath{\mathbf{w}}}
\newcommand{\Bx}{\ensuremath{\mathbf{x}}}
\newcommand{\By}{\ensuremath{\mathbf{y}}}
\newcommand{\Bz}{\ensuremath{\mathbf{z}}}
\newcommand{\BA}{\ensuremath{\boldsymbol{A}}}
\newcommand{\BB}{\ensuremath{\boldsymbol{B}}}
\newcommand{\BBbar}{\ensuremath{\bar{\boldsymbol{B}}}}
\newcommand{\BC}{\ensuremath{\boldsymbol{C}}}
\newcommand{\BCbar}{\ensuremath{\bar{\boldsymbol{C}}}}
\newcommand{\Cbar}{\ensuremath{\bar{C}}}
\newcommand{\BD}{\ensuremath{\boldsymbol{D}}}
\newcommand{\BE}{\ensuremath{\boldsymbol{E}}}
\newcommand{\BF}{\ensuremath{\boldsymbol{F}}}
\newcommand{\BFbar}{\ensuremath{\bar{\boldsymbol{F}}}}
\newcommand{\Fbar}{\ensuremath{\bar{F}}}
\newcommand{\BG}{\ensuremath{\boldsymbol{G}}}
\newcommand{\BH}{\ensuremath{\boldsymbol{H}}}
\newcommand{\BI}{\ensuremath{\boldsymbol{I}}}
\newcommand{\BJ}{\ensuremath{\boldsymbol{J}}}
\newcommand{\BK}{\ensuremath{\boldsymbol{K}}}
\newcommand{\BL}{\ensuremath{\boldsymbol{L}}}
\newcommand{\BM}{\ensuremath{\boldsymbol{M}}}
\newcommand{\BNv}{\ensuremath{\mathbf{N}}}
\newcommand{\BN}{\ensuremath{\boldsymbol{N}}}
\newcommand{\BP}{\ensuremath{\boldsymbol{P}}}
\newcommand{\BQ}{\ensuremath{\boldsymbol{Q}}}
\newcommand{\BR}{\ensuremath{\boldsymbol{R}}}
\newcommand{\BS}{\ensuremath{\boldsymbol{S}}}
\newcommand{\BT}{\ensuremath{\boldsymbol{T}}}
\newcommand{\BTv}{\ensuremath{\mathbf{T}}}
\newcommand{\BW}{\ensuremath{\boldsymbol{W}}}
\newcommand{\BX}{\ensuremath{\mathbf{X}}}
\newcommand{\BXT}{\ensuremath{\boldsymbol{X}}}
\newcommand{\BY}{\ensuremath{\boldsymbol{Y}}}
\newcommand{\Trial}{\ensuremath{\text{trial}}}
\newcommand{\Tint}{\ensuremath{\text{int}}}
\newcommand{\Text}{\ensuremath{\text{ext}}}
\newcommand{\Tkin}{\ensuremath{\text{kin}}}
\newcommand{\Tbody}{\ensuremath{\text{body}}}
\newcommand{\Tmin}{\ensuremath{\text{min}}}
\newcommand{\Tmax}{\ensuremath{\text{max}}}
\newcommand{\Tor}{\ensuremath{\text{or}}}
\newcommand{\Tr}{\ensuremath{\text{tr}}}
\newcommand{\Tdev}{\ensuremath{\text{dev}}}
\newcommand{\Tvol}{\ensuremath{\text{vol}}}
\newcommand{\Dev}{\ensuremath{\text{dev}}}
\newcommand{\Half}{\ensuremath{\tfrac{1}{2}}}
\newcommand{\SThr}{\ensuremath{\sqrt{3}}}
\newcommand{\STT}{\ensuremath{\frac{\sqrt{3}}{2}}}
\newcommand{\Third}{\ensuremath{\tfrac{1}{3}}}
\newcommand{\TwoThird}{\ensuremath{\tfrac{2}{3}}}
\newcommand{\Inner}[2]{\ensuremath{\langle#1,~#2\rangle}}
\newcommand{\Bcross}[2]{\ensuremath{#1\boldsymbol{\times}#2}}
\newcommand{\Bdot}[2]{\ensuremath{#1\cdot#2}}
\newcommand{\Dyad}[2]{\ensuremath{#1\boldsymbol{\otimes}#2}}
\newcommand{\Grad}[1]{\ensuremath{\Bnabla #1}}
\newcommand{\Gradp}[1]{\ensuremath{\Bnabla' #1}}
\newcommand{\Grads}[1]{\ensuremath{\Bnabla_s #1}}
\newcommand{\Grady}[1]{\ensuremath{\Bnabla_y #1}}
\newcommand{\Lap}[1]{\ensuremath{\nabla^2 #1}}
\newcommand{\Biharm}[1]{\ensuremath{\nabla^4 #1}}
\newcommand{\Div}[1]{\ensuremath{\Bdot{\Bnabla}{#1}}}
\newcommand{\Divp}[1]{\ensuremath{\Bdot{\Bnabla'}{#1}}}
\newcommand{\Divy}[1]{\ensuremath{\Bdot{\Bnabla_y}{#1}}}
\newcommand{\Curl}[1]{\ensuremath{\Bcross{\Bnabla}{#1}}}
\newcommand{\Curlp}[1]{\ensuremath{\Bcross{\Bnabla'}{#1}}}
\newcommand{\Curls}[1]{\ensuremath{\Bcross{\Bnabla_s}{#1}}}
\newcommand{\Curly}[1]{\ensuremath{\Bcross{\Bnabla_y}{#1}}}
\newcommand{\Gradu}{\ensuremath{\Grad{\Bu}}}
\newcommand{\Divu}{\ensuremath{\Div{\Bu}}}
\newcommand{\Curlu}{\ensuremath{\Curl{\Bu}}}
\newcommand{\Gradv}{\ensuremath{\Grad{\Bv}}}
\newcommand{\Divv}{\ensuremath{\Div{\Bv}}}
\newcommand{\Curlv}{\ensuremath{\Curl{\Bv}}}
\newcommand{\Dualn}{\ensuremath{\Bdual{\Bn}{\Bn}}}
\newcommand{\Over}[1]{\ensuremath{\frac{1}{#1}}}
\newcommand{\Diff}[2]{\ensuremath{\frac{d #1}{d #2}}}
\newcommand{\Partial}[2]{\ensuremath{\frac{\displaystyle\partial #1}{\displaystyle\partial #2}}}
\newcommand{\PPartial}[2]{\ensuremath{\frac{\partial^2 #1}{\partial #2^2}}}
\newcommand{\PPartialA}[3]{\ensuremath{\frac{\partial^2 #1}{\partial #2\partial#3}}}
\newcommand{\FPartial}[2]{\ensuremath{\frac{\partial^4 #1}{\partial #2^4}}}
\newcommand{\FPartialA}[3]{\ensuremath{\frac{\partial^4 #1}{\partial #2^2
         \partial #3^2}}}
\newcommand{\DotMbT}{\ensuremath{\dot{\MbT}}}
\newcommand{\TildeMbT}{\ensuremath{\widetilde{\MbT}}}
\newcommand{\BarT}{\ensuremath{\overline{T}}}
\newcommand{\Barq}{\ensuremath{\overline{q}}}
\newcommand{\Domega}{\ensuremath{\partial{\Omega}}}
\newcommand{\Av}[1]{\ensuremath{\left\langle#1\right\rangle}}
\newcommand{\AvSig}{\ensuremath{\langle\Bsig\rangle}}
\newcommand{\AvTau}{\ensuremath{\langle\Btau\rangle}}
\newcommand{\AvP}{\ensuremath{\langle\BP\rangle}}
\newcommand{\AvEps}{\ensuremath{\langle\Beps\rangle}}
\newcommand{\AvEpsdot}{\ensuremath{\langle\dot{\Beps}\rangle}}
\newcommand{\AvDisp}{\ensuremath{\langle\Bu\rangle}}
\newcommand{\AvF}{\ensuremath{\langle\BF\rangle}}
\newcommand{\AvFdot}{\ensuremath{\langle\dot{\BF}\rangle}}
\newcommand{\Avl}{\ensuremath{\overline{\Bl}}}
\newcommand{\AvSigBar}{\ensuremath{\overline{\Bsig}}}
\newcommand{\AvTauBar}{\ensuremath{\overline{\Btau}}}
\newcommand{\AvOmega}{\ensuremath{\langle\Bomega\rangle}}
\newcommand{\AvGradu}{\ensuremath{\langle\Gradu\rangle}}
\newcommand{\AvGradudot}{\ensuremath{\langle\Grad{\dot{\Bu}}\rangle}}
\newcommand{\AvGradv}{\ensuremath{\langle\Gradv\rangle}}
\newcommand{\AvPower}{\ensuremath{\langle\Bsig:\Gradv\rangle}}
\newcommand{\AvPowerInf}{\ensuremath{\langle\Bsig:\dot{\Beps}\rangle}}
\newcommand{\AvWorkInf}{\ensuremath{\langle\Bsig:\Beps\rangle}}
\newcommand{\AvPowerPF}{\ensuremath{\langle\BP^T:\dot{\BF}\rangle}}
\newcommand{\DA}{\ensuremath{\text{dA}}}
\newcommand{\DAvec}{\ensuremath{\text{d}\mathbf{A}}}
\newcommand{\Da}{\ensuremath{\text{da}}}
\newcommand{\Davec}{\ensuremath{\text{d}\mathbf{a}}}
\newcommand{\DV}{\ensuremath{\text{dV}}}
\newcommand{\BCe}{\ensuremath{\mathcal{E}}}
\newcommand{\GradX}[1]{\ensuremath{\Bnabla_0~#1}}
\newcommand{\DivX}[1]{\ensuremath{\Bdot{\Bnabla_0}{#1}}}
\newcommand{\Bxdot}{\ensuremath{\dot{\Bx}}}
\newcommand{\BFdot}{\ensuremath{\dot{\BF}}}
\newcommand{\BAv}{\ensuremath{\mathbf{A}}}
\newcommand{\BBv}{\ensuremath{\mathbf{B}}}
\newcommand{\BDv}{\ensuremath{\mathbf{D}}}
\newcommand{\BEv}{\ensuremath{\mathbf{E}}}
\newcommand{\BFv}{\ensuremath{\mathbf{F}}}
\newcommand{\BHv}{\ensuremath{\mathbf{H}}}
\newcommand{\BJv}{\ensuremath{\mathbf{J}}}
\newcommand{\BMv}{\ensuremath{\mathbf{M}}}
\newcommand{\BPv}{\ensuremath{\mathbf{P}}}
\newcommand{\BRv}{\ensuremath{\mathbf{R}}}
\newcommand{\BVv}{\ensuremath{\mathbf{V}}}
\newcommand{\Bdelta}{\ensuremath{\boldsymbol{\delta}}}
\newcommand{\Beps}{\ensuremath{\boldsymbol{\epsilon}}}
\newcommand{\BVeps}{\ensuremath{\boldsymbol{\varepsilon}}}
\newcommand{\BDtildev}{\ensuremath{\mathbf{\widetilde{D}}}}
\newcommand{\IntInfT}{\ensuremath{\int_{-\infty}^t}}
\newcommand{\IntInfInf}{\ensuremath{\int_{-\infty}^{\infty}}}
\newcommand{\IntInfZero}{\ensuremath{\int_{-\infty}^{0}}}
\newcommand{\IntZeroInf}{\ensuremath{\int_{0}^{\infty}}}
\newcommand{\IIntInfInf}{\ensuremath{\int_{-\infty}^{\infty}\int_{-\infty}^{\infty}}}
\newcommand{\IIIntInfInf}{\ensuremath{\int_{-\infty}^{\infty}\int_{-\infty}^{\infty}\int_{-\infty}^{\infty}}}
\newcommand{\Dtau}{\ensuremath{\text{d}\tau}}
\newcommand{\domega}{\ensuremath{\text{d}\omega}}
\newcommand{\dOmega}{\ensuremath{\text{d}\Omega}}
\newcommand{\dGamma}{\ensuremath{\text{d}\Gamma}}
\newcommand{\dzeta}{\ensuremath{\text{d}\zeta}}
\newcommand{\Ds}{\ensuremath{\text{d}s}}
\newcommand{\Dt}{\ensuremath{\text{d}t}}
\newcommand{\Dx}{\ensuremath{\text{d}\Bx}}
\newcommand{\dr}{\ensuremath{\text{d}r}}
\newcommand{\dx}{\ensuremath{\text{d}x}}
\newcommand{\dy}{\ensuremath{\text{d}y}}
\newcommand{\dz}{\ensuremath{\text{d}z}}
\newcommand{\dk}{\ensuremath{\text{d}k}}
\newcommand{\dBx}{\ensuremath{\text{d}\Bx}}
\newcommand{\dBk}{\ensuremath{\text{d}\Bk}}
\newcommand{\dBr}{\ensuremath{\text{d}\Br}}
\newcommand{\BKbar}{\ensuremath{\boldsymbol{\bar{K}}}}
\newcommand{\That}{\ensuremath{\widehat{T}}}
\newcommand{\Bahat}{\ensuremath{\widehat{\Ba}}}
\newcommand{\Bbhat}{\ensuremath{\widehat{\Bb}}}
\newcommand{\BAhat}{\ensuremath{\widehat{\BA}}}
\newcommand{\BBhat}{\ensuremath{\widehat{\BB}}}
\newcommand{\BBhatv}{\ensuremath{\widehat{\mathbf{B}}}}
\newcommand{\BDhatv}{\ensuremath{\widehat{\mathbf{D}}}}
\newcommand{\BEhatv}{\ensuremath{\widehat{\mathbf{E}}}}
\newcommand{\BFhatv}{\ensuremath{\widehat{\mathbf{F}}}}
\newcommand{\BHhatv}{\ensuremath{\widehat{\mathbf{H}}}}
\newcommand{\BPhatv}{\ensuremath{\widehat{\mathbf{P}}}}
\newcommand{\BVhatv}{\ensuremath{\widehat{\mathbf{V}}}}
\newcommand{\Rea}{\ensuremath{\text{Re}}}
\newcommand{\Img}{\ensuremath{\text{Im}}}
\newcommand{\Teff}{\ensuremath{\text{eff}}}
\newcommand{\Tand}{\ensuremath{\text{and}}}
\newcommand{\CalE}{\ensuremath{\mathcal{E}}}
\newcommand{\CalH}{\ensuremath{\mathcal{H}}}
\newcommand{\CalJ}{\ensuremath{\mathcal{J}}}
\newcommand{\CalU}{\ensuremath{\mathcal{U}}}
\newcommand{\Dhat}{\ensuremath{\widehat{D}}}
\newcommand{\Ehat}{\ensuremath{\widehat{E}}}
\newcommand{\Fhat}{\ensuremath{\widehat{F}}}
\newcommand{\Phat}{\ensuremath{\widehat{P}}}
\newcommand{\Uhat}{\ensuremath{\widehat{U}}}
\newcommand{\Vhat}{\ensuremath{\widehat{V}}}
\newcommand{\fhat}{\ensuremath{\widehat{f}}}
\newcommand{\ghat}{\ensuremath{\widehat{g}}}
\newcommand{\phat}{\ensuremath{\widehat{p}}}
\newcommand{\uhat}{\ensuremath{\widehat{u}}}
\newcommand{\vhat}{\ensuremath{\widehat{v}}}
\newcommand{\xhat}{\ensuremath{\widehat{x}}}
\newcommand{\yhat}{\ensuremath{\widehat{y}}}
\newcommand{\Beq}{\begin{equation}}
\newcommand{\Eeq}{\end{equation}}
\newcommand{\Bal}{\begin{aligned}}
\newcommand{\Eal}{\end{aligned}}
\newcommand{\Ibar}{\ensuremath{\bar{I}}}
\newcommand{\Tbar}{\ensuremath{\text{bar}}}
\newcommand{\Tball}{\ensuremath{\text{ball}}}
\newcommand{\Buhat}{\ensuremath{\widehat{\Bu}}}
\newcommand{\BHhat}{\ensuremath{\widehat{\BH}}}
\newcommand{\Bsighat}{\ensuremath{\widehat{\Bsig}}}
\newcommand{\Bepshat}{\ensuremath{\widehat{\Beps}}}
\newcommand{\sighat}{\ensuremath{\widehat{\sigma}}}
\newcommand{\epshat}{\ensuremath{\widehat{\epsilon}}}
\newcommand{\rhohat}{\ensuremath{\widehat{\rho}}}
\newcommand{\phihat}{\ensuremath{\widehat{\varphi}}}
\newcommand{\kappahat}{\ensuremath{\widehat{\kappa}}}
\newcommand{\SfK}{\ensuremath{\boldsymbol{\mathsf{K}}}}
\newcommand{\SfZero}{\ensuremath{\boldsymbol{\mathsf{0}}}}
\newcommand{\Gradbar}[1]{\ensuremath{\overline{\Bnabla} #1}}
\newcommand{\Divbar}[1]{\ensuremath{\Bdot{\overline{\Bnabla}}{#1}}}
\newcommand{\ktilde}{\ensuremath{\widetilde{k}}}
\newcommand{\Etilde}{\ensuremath{\widetilde{E}}}
\newcommand{\Htilde}{\ensuremath{\widetilde{H}}}
\newcommand{\Ktilde}{\ensuremath{\widetilde{K}}}
\newcommand{\Rtilde}{\ensuremath{\widetilde{R}}}
\newcommand{\Ttilde}{\ensuremath{\widetilde{T}}}
\newcommand{\TAi}{\ensuremath{\text{Ai}}}
\newcommand{\TBi}{\ensuremath{\text{Bi}}}
\newcommand{\sgn}{\ensuremath{\text{sgn}}}
\newcommand{\DelTwo}{\ensuremath{\Delta/2}}
\newcommand{\Bepseff}{\ensuremath{\Beps_\Teff}}
\newcommand{\Bmueff}{\ensuremath{\Bmu_\Teff}}
\newcommand{\epseff}{\ensuremath{\epsilon_\Teff}}
\newcommand{\mueff}{\ensuremath{\mu_\Teff}}
\newcommand{\BAeff}{\ensuremath{\BA_\Teff}}
\newcommand{\Conj}[1]{\ensuremath{\overline{#1}}}
\newcommand{\kappatilde}{\ensuremath{\widetilde{\kappa}}}
\newcommand{\lambdatilde}{\ensuremath{\widetilde{\lambda}}}

\newcommand{\MAmat}{\ensuremath{\uuline{\mathsf{A}}}}
\newcommand{\MBmat}{\ensuremath{\uuline{\mathsf{B}}}}
\newcommand{\MCmat}{\ensuremath{\uuline{\mathsf{C}}}}
\newcommand{\Ma}{\ensuremath{\uuline{\mathsf{a}}}}
\newcommand{\Matilde}{\ensuremath{\widetilde{\Ma}}}
\newcommand{\Mb}{\ensuremath{\uuline{\mathsf{b}}}}
\newcommand{\Mf}{\ensuremath{\uuline{\mathsf{f}}}}
\newcommand{\Mg}{\ensuremath{\uuline{\mathsf{g}}}}
\newcommand{\Mn}{\ensuremath{\uuline{\mathsf{n}}}}
\newcommand{\Mntilde}{\ensuremath{\widetilde{\Mn}}}
\newcommand{\Mp}{\ensuremath{\uuline{\mathsf{p}}}}
\newcommand{\Mq}{\ensuremath{\uuline{\mathsf{q}}}}
\newcommand{\Mr}{\ensuremath{\uuline{\mathsf{r}}}}
\newcommand{\Ms}{\ensuremath{\uuline{\mathsf{s}}}}
\newcommand{\Mnhat}{\ensuremath{\uuline{\widehat{\mathsf{n}}}}}
\newcommand{\Mqhat}{\ensuremath{\uuline{\widehat{\mathsf{q}}}}}
\newcommand{\Mshat}{\ensuremath{\uuline{\widehat{\mathsf{s}}}}}
\newcommand{\Mu}{\ensuremath{\uuline{\mathsf{u}}}}
\newcommand{\Mv}{\ensuremath{\uuline{\mathsf{v}}}}
\newcommand{\Mw}{\ensuremath{\uuline{\mathsf{w}}}}
\newcommand{\Mx}{\ensuremath{\uuline{\mathsf{x}}}}
\newcommand{\Mone}{\ensuremath{\uuline{\mathsf{1}}}}
\newcommand{\Mzero}{\ensuremath{\uuline{\mathsf{0}}}}

\newcommand{\erf}{\text{erf}}
\newcommand{\Xidot}{\ensuremath{\dot{\xi}}}
\newcommand{\BU}{\ensuremath{\boldsymbol{U}}}
\newcommand{\Bbeta}{\ensuremath{\boldsymbol{\beta}}}
\newcommand{\SfB}{\ensuremath{\boldsymbol{\mathsf{B}}}}



% For adding "DRAFT" to each page uncomment the following
% --------
\usepackage{eso-pic}
\makeatletter
\AddToShipoutPicture{%
  \setlength{\@tempdimb}{.5\paperwidth}%
  \setlength{\@tempdimc}{.5\paperheight}%
  \setlength{\unitlength}{1pt}%
  \put(\strip@pt\@tempdimb,\strip@pt\@tempdimc){%
    \makebox(0,0){\rotatebox{45}{\textcolor[gray]{0.9}%
      {\fontsize{6cm}{6cm}\selectfont{DRAFT}}}}%
  }%
}
\makeatother
% --------

\begin{document}
\DeclareGraphicsExtensions{.jpg,.pdf}

\title{Cam-Clay model based on Borja et al. 1997}
\author{Biswajit Banerjee}
\maketitle
\tableofcontents
\newpage

\section{Introduction}
Introduce the equations and how they differ from Fossum-Brannon.

\section{Quantities that are needed in a Uintah implementation}
\subsection{Elasticity}
The elastic strain energy density in Borja's model has the form
\[
  W(\Ve^e_v,\Ve^e_s) = W_\Tvol(\Ve^e_v) + W_\Tdev(\Ve^e_v, Ve^e_s)
\]
where
\[
   \Bal
    W_\Tvol(\Ve^e_v) & = -p_0\kappatilde\,\exp\left(-\frac{\Ve^e_v - \Ve^e_{v0}}{\kappatilde}\right) \\
    W_\Tdev(\Ve^e_v,\Ve^e_s) & =  \tfrac{3}{2}\,\mu\,(\Ve^e_s)^2
   \Eal
\]
where $\Ve^e_{v0}$ is the volumetric strain corresponding to a mean normal compressive stress $p_0$ 
(tension positive), $\kappatilde$ is the elastic compressibility index, and the shear modulus is given by
\[
  \mu = \mu_0 + \frac{\alpha}{\kappatilde}\,W_\Tvol(\Ve^e_v) 
      = \mu_0 - \alpha p_0\,\exp\left(-\frac{\Ve^e_v - \Ve^e_{v0}}{\kappatilde}\right) 
      = \mu_0 - \mu_\Tvol\,.
\]
The parameter $\alpha$ determines the extent of coupling between the volumetric and deviatoric 
responses.  For consistency with isotropic elasticity, Rebecca Brannon suggests that $\alpha = 0$ (citation?). 

The stress invariants $p$ and $q$ are defined as
\[
  \Bal
    p &= \Partial{W}{\Ve^e_v} = p_0\left[1 + \tfrac{3}{2}\,\frac{\alpha}{\kappatilde}\,(\Ve^e_s)^2\right]
         \exp\left(-\frac{\Ve^e_v - \Ve^e_{v0}}{\kappatilde}\right) 
       = p_0\,\beta\,\exp\left(-\frac{\Ve^e_v - \Ve^e_{v0}}{\kappatilde}\right) \\
    q &= \Partial{W}{\Ve^e_s} = 3\left[\mu_0 - \alpha p_0\exp\left(-\frac{\Ve^e_v - \Ve^e_{v0}}{\kappatilde}\right) 
         \right]\Ve^e_s  = 3\mu\,\Ve^e_s\,.
  \Eal
\]
The derivatives of the stress invariants are
\[
  \Bal
    \Partial{p}{\Ve^e_v} & = -\frac{p_0}{\kappatilde}
         \left[1 + \tfrac{3}{2}\,\frac{\alpha}{\kappatilde}\,(\Ve^e_s)^2\right]
         \,\exp\left(-\frac{\Ve^e_v - \Ve^e_{v0}}{\kappatilde}\right) 
        = -\frac{p}{\kappatilde} \\
    \Partial{p}{\Ve^e_s} & = \Partial{q}{\Ve^e_v} = \frac{3\alpha p_0 \Ve^e_s}{\kappatilde}
         \,\exp\left(-\frac{\Ve^e_v - \Ve^e_{v0}}{\kappatilde}\right) = \frac{3\alpha p}{\beta \kappatilde}\,\Ve^e_s 
         = \frac{3\mu_\Tvol}{\kappatilde}\,\Ve^e_s\\
    \Partial{q}{\Ve^e_s} & = 3 \left[\mu_0 - \alpha p_0
         \,\exp\left(-\frac{\Ve^e_v - \Ve^e_{v0}}{\kappatilde}\right) \right] = 3 \mu\,.
  \Eal
\]

\subsection{Plasticity}
For plasticity we use a Cam-Clay yield function of the form
\[
   f = \left(\frac{q}{M}\right)^2 + p(p-p_c) 
\]
where $M$ is the slope of the critical state line and the consolidation pressure $p_c$ is an internal variable that 
evolves according to 
\[
   \frac{1}{p_c}\,\Deriv{p_c}{t} = \frac{1}{\lambdatilde - \kappatilde}\,\Deriv{\Ve^p_v}{t} \,.
\]
The derivatives of $f$ that are of interest are
\[
   \Bal
     \Partial{f}{p} & = 2p - p_c \\
     \Partial{f}{q} & = \frac{2q}{M^2} \,.
   \Eal
\]
If we integrate the equation for $p_c$ from $t_{n}$ to $t_{n+1}$, we can show that
\[
   (p_c)_{n+1} = (p_c)_n \exp\left[\frac{(\Ve_v^e)_\Trial - (\Ve_v^e)_{n+1}}{\lambdatilde - \kappatilde}\right] \,.
\]
The derivative of $p_c$ that is of interest is
\[
   \Partial{p_c}{(\Ve_v^e)_{n+1}} = -\frac{(p_c)_n}{\lambdatilde-\kappatilde}\,\exp\left[\frac{(\Ve^e_v)_\Trial - (\Ve_v^e)_{n+1}}{\lambdatilde-\kappatilde}\right] \,.
\]

\section{Why these quantities are needed: stress update based Rich Reguiero's notes}
The volumetric and deviatoric components of the elastic strain $\Beps^e$ are defined
as follows:
\[
   \BeT^e = \Beps^e - \tfrac{1}{3}\Ve^e_v\,\Bone = \Beps^e - \tfrac{1}{3}\Tr(\Beps^e)\,\Bone
   \quad \Tand \quad
   \Ve^e_s = \sqrt{\tfrac{2}{3}}\Norm{\BeT^e}{}  = \sqrt{\tfrac{2}{3}}\sqrt{\BeT^e:\BeT^e} \,.
\]
The stress tensor is decomposed into a volumetric and a deviatoric component
\[
   \Bsig = p\,\Bone + \sqrt{\tfrac{2}{3}}\, q\, \BnT \quad \text{with} \quad
   \BnT = \cfrac{\BeT^e}{\Norm{\BeT^e}{}} = \sqrt{\tfrac{2}{3}}\, \cfrac{\BeT^e}{\Ve^e_s} \,.
\]
The models used to determine $p$ and $q$ are
\[
  \Bal
    p &= p_0\beta\exp\left[-\frac{\Ve^e_v - \Ve^e_{v0}}{\kappatilde}\right] \quad \text{with} \quad
     \beta = 1 + \tfrac{3}{2}\,\frac{\alpha}{\kappatilde}\,(\Ve^e_s)^2 \\
    q &= 3\mu\Ve^e_s \,.
  \Eal
\]
The strains are updated using
\[
  \Beps^e = \Beps^e_{\rm trial} - \Delta\gamma\,\Partial{f}{\Bsig}
  \quad \text{where} \quad \Beps^e_{\rm trial} = \Beps^e_n + \Delta\Beps
     = \Beps^e_n + (\Beps - \Beps_n) \,.
\]

{\footnotesize
{\bf Remark 1:}  The interface with MPMICE, among other things in Uintah, requires the 
computation of the quantity $dp/dJ$.  Since $J$ does not appear in the above equation we
proceed as explained below.
\[
  \Bal
   J & = \det(\BF) = \det(\Bone + \GradX{\Bu}) = \det(\Bone + \Beps) \\
     & = 1 + \Tr\Beps + \Half\left[(\Tr\Beps)^2 - \Tr(\Beps^2)\right] + \det(\Beps) \,.
     & = 1 + \Ve_v + \Half\left[\Ve_v^2 - \Tr(\Beps^2)\right] + \det(\Beps) \,.
  \Eal
\]
Also,
\[
   J = \frac{\rho_0}{\rho} = \frac{V}{V_0} \quad \Tand \quad
   \Ve_v = \frac{V-V_0}{V_0} = \frac{V}{V_0} - 1 = J - 1 \,.
\]
We use the relation $J = 1 + \Ve_v$ while keeping in mind that this is {\em true only for
infinitesimal strains and plastic incompressibility} for which 
$\Ve_v^2$, $\Tr(\Beps^2)$, and $\det(\Beps)$ are zero.  Under these conditions
\[
   \Partial{p}{J} = \Partial{p}{\Ve_v}\,\Partial{\Ve_v}{J} = \Partial{p}{\Ve_v} \quad \Tand 
   \quad
   \Partial{p}{\rho} = \Partial{p}{\Ve_v}\,\Partial{\Ve_v}{J}\,\Partial{J}{\rho} 
      = -\frac{J}{\rho}\,\Partial{p}{\Ve_v} \,.
\]

{\bf Remark 2:} MPMICE also needs the density at a given pressure.  For the Borja model, with
$\Ve_v = J-1 = \rho_0/\rho -1$, we have
\[
  \rho = \rho_0\left[1 + \Ve_{v0} + \kappatilde\ln\left(\frac{p}{p_0\beta}\right)\right]^{-1}\,.
\]

{\bf Remark 3:}  The quantity $q$ is related to the deviatoric part of the Cauchy stress, $\BsT$
as follows:
\[
   q = \sqrt{3J_2} \quad \text{where} \quad J_2 = \Half\,\BsT:\BsT \,.
\]
The shear modulus relates the deviatoric stress $\BsT$ to the deviatoric strain $\BeT^e$.  We
assume a relation of the form
\[
   \BsT = 2\mu\BeT^e \,.
\]
Note that the above relation assumes a linear elastic type behavior.  Then we get the Borja 
shear model:
\[
  q = \sqrt{\tfrac{3}{2}\,\BsT:\BsT} = \sqrt{\tfrac{3}{2}}\,(2\mu)\,\sqrt{\BeT^e:\BeT^e}
     = \sqrt{\tfrac{3}{2}}\,(2\mu)\,\sqrt{\tfrac{3}{2}}\,\Ve^e_s = 3\mu\Ve^e_s \,.
\]
}



\subsection{Elastic-plastic stress update}
For elasto-plasticity we start with a yield function of the form
\[
   f = \left(\frac{q}{M}\right)^2 + p(p-p_c) \le 0 
   \quad \text{where} \quad
  \frac{1}{p_c}\,\Deriv{p_c}{t} = \frac{1}{\lambdatilde- \kappatilde}\,\Deriv{\Ve_v^p}{t} \,.
\]
Integrating the ODE for $p_c$ with the initial condition $p_c(t_n) = (p_c)_n$, at $t = t_{n+1}$, 
\[
   (p_c)_{n+1} = (p_c)_n \exp\left[\frac{(\Ve_v^p)_{n+1} - (\Ve_v^p)_n}{\lambdatilde - \kappatilde}\right] \,.
\]
From the additive decomposition of the strain into elastic and plastic parts, and if the elastic trial 
strain is defined as 
\[
   (\Ve_v^e)_\Trial := (\Ve_v^e)_n + \Delta\Ve_v
\]
we have
\[
   \Ve_v^p = \Ve_v - \Ve_v^e \quad \implies \quad
   (\Ve_v^p)_{n+1} - (\Ve_v^p)_n = (\Ve_v)_{n+1} - (\Ve_v^e)_{n+1} - (\Ve_v)_{n} + (\Ve_v^e)_{n} 
                               = \Delta\Ve_v + (\Ve_v^e)_{n} - (\Ve_v^e)_{n+1} 
                               = (\Ve_v^e)_\Trial - (\Ve_v^e)_{n+1} \,.
\]
Therefore we can write
\[
   (p_c)_{n+1} = (p_c)_n \exp\left[\frac{(\Ve_v^e)_\Trial - (\Ve_v^e)_{n+1}}{\lambdatilde - \kappatilde}\right] \,.
\]
The flow rule is assumed to be given by
\[
   \Partial{\Beps^p}{t} = \gamma\,\Partial{f}{\Bsig} \,.
\]
Integration of the PDE with backward Euler gives
\[
  \Beps^p_{n+1} = \Beps^p_n + \Delta t\,\gamma_{n+1}\,\left[\Partial{f}{\Bsig}\right]_{n+1} 
             = \Beps^p_n + \Delta\gamma\,\left[\Partial{f}{\Bsig}\right]_{n+1} \,.
\]
This equation can be expressed in terms of the trial elastic strain as follows.
\[
  \Beps_{n+1}-\Beps^e_{n+1} = \Beps_n - \Beps^e_n + \Delta\gamma\,\left[\Partial{f}{\Bsig}\right]_{n+1} 
\]
or
\[
  \Beps^e_{n+1} = \Delta\Beps + \Beps^e_n - \Delta\gamma\,\left[\Partial{f}{\Bsig}\right]_{n+1} 
             = \Beps^e_\Trial - \Delta\gamma\,\left[\Partial{f}{\Bsig}\right]_{n+1} \,.
\]
In terms of the volumetric and deviatoric components
\[
  (\Ve_v^e)_{n+1} = \Tr(\Beps^e_{n+1}) = \Tr(\Beps^e_\Trial) - \Delta\gamma\,\Tr\left[\Partial{f}{\Bsig}\right]_{n+1} = (\Ve_v^e)_\Trial - \Delta\gamma\,\Tr\left[\Partial{f}{\Bsig}\right]_{n+1} 
\]
and
\[
  \BeT^e_{n+1} = \BeT^e_\Trial - \Delta\gamma\,\left[\left(\Partial{f}{\Bsig}\right)_{n+1}
      - \tfrac{1}{3}\,\Tr\left(\Partial{f}{\Bsig}\right)_{n+1}\Bone\right] \,.
\]
With $\BsT = \Bsig - p\Bone$, we have
\[
  \Partial{f}{\Bsig} = \Partial{f}{\BsT}:\Partial{\BsT}{\Bsig} + \Partial{f}{p}\,\Partial{p}{\Bsig}
    = \Partial{f}{\BsT}:[\SfI^{(s)} - \tfrac{1}{3}\,\Bone\otimes\Bone] 
      + \Partial{f}{p}\,\Bone
    = \Partial{f}{\BsT} - \tfrac{1}{3}\,\Tr\left[\Partial{f}{\BsT}\right]\Bone
      + \Partial{f}{p}\,\Bone
\]
and
\[
  \tfrac{1}{3}\,\Tr\left[\Partial{f}{\Bsig}\right]\Bone = 
    \tfrac{1}{3}\left(\Tr\left[\Partial{f}{\BsT}\right] - \Tr\left[\Partial{f}{\BsT}\right]
      + 3\Partial{f}{p}\right)\Bone  =  \Partial{f}{p}\,\Bone \,.
\]

{\footnotesize
{\bf Remark 4:}  Note that, because $\Bsig = \Bsig(p, q, p_c)$ the 
chain rule should contain a contribution from $p_c$:
\[
  \Partial{f}{\Bsig} = \Partial{f}{q}\,\Partial{q}{\Bsig} + \Partial{f}{p}\,\Partial{p}{\Bsig}
                       + \Partial{f}{p_c}\,\Partial{p_c}{\Bsig} \,.
\]
However, the Borja implementation does not consider that extra term.  Also note that for the present model
\[
  \Bsig = \Bsig(p(\Ve^e_v,\Ve^e_s,\Ve^p_v, \Ve^p_s), \BsT(\Ve^e_v, \Ve^e_s, \Ve^p_v,\Ve^p_s), p_c(\Ve^p_v))
\]
}

Therefore, for situations where $\Tr(\partial f/\partial \BsT) = \Bzero$, we have
\[
  \Partial{f}{\Bsig} - \tfrac{1}{3}\,\Tr\left[\Partial{f}{\Bsig}\right]\Bone = 
     \Partial{f}{\BsT} - \tfrac{1}{3}\,\Tr\left[\Partial{f}{\BsT}\right]\Bone =
     \Partial{f}{\BsT} \,.
\]
The deviatoric strain update can be written as
\[
  \BeT^e_{n+1} = \BeT^e_\Trial - \Delta\gamma\,\left(\Partial{f}{\BsT}\right)_{n+1} 
\]
and the shear invariant update is
\[
  (\Ve_s^e)_{n+1} = \sqrt{\tfrac{2}{3}}\,
  \sqrt{\BeT^e_{n+1}:\BeT^e_{n+1}}
  = \sqrt{\tfrac{2}{3}}\,\sqrt{\BeT^e_\Trial:\BeT^e_\Trial 
     - 2\Delta\gamma\,\left[\Partial{f}{\BsT}\right]_{n+1}:\BeT^e_\Trial
     + (\Delta\gamma)^2\left[\Partial{f}{\BsT}\right]_{n+1}:\left[\Partial{f}{\BsT}\right]_{n+1}}
\]
The derivative of $f$ can be found using the chain rule (for smooth $f$):
\[
   \Partial{f}{\Bsig} = \Partial{f}{p}\,\Partial{p}{\Bsig} + \Partial{f}{q}\,\Partial{q}{\Bsig}
     = (2p - p_c)\,\Partial{p}{\Bsig} + \frac{2q}{M^2}\,\Partial{q}{\Bsig} \,.
\]
Now, with $p = 1/3\,\Tr(\Bsig)$ and $q = \sqrt{3/2\,\Bs:\Bs}$, we have
\[
  \Bal
   \Partial{p}{\Bsig} & = \Partial{}{\Bsig}\left[\tfrac{1}{3}\,\Tr(\Bsig)\right] = \tfrac{1}{3}\,\Bone \\
   \Partial{q}{\Bsig} & = \Partial{}{\Bsig}\left[\sqrt{\tfrac{3}{2}\,\BsT:\BsT}\right]
     = \sqrt{\tfrac{3}{2}}\,\frac{1}{\sqrt{\BsT:\BsT}}\,\Partial{\BsT}{\Bsig}:\BsT
     = \sqrt{\tfrac{3}{2}}\,\frac{1}{\Norm{\BsT}{}}\,\left[\SfI^{(s)}-\tfrac{1}{3}\Bone\otimes\Bone\right]:\BsT
     = \sqrt{\tfrac{3}{2}}\,\frac{\BsT}{\Norm{\BsT}{}}\,.
  \Eal
\]
Therefore,
\[
   \Partial{f}{\Bsig} = \frac{2p - p_c}{3}\,\Bone + \sqrt{\tfrac{3}{2}}\,\frac{2q}{M^2}\,\frac{\BsT}{\Norm{\BsT}{}} \,.
\]
Recall that
\[
   \Bsig = p\,\Bone + \sqrt{\tfrac{2}{3}}\, q\, \BnT = p\,\Bone + \BsT \,.
\]
Therefore,
\[
   \BsT = \sqrt{\tfrac{2}{3}}\,q\,\BnT 
   \quad \Tand \quad
   \Norm{\BsT}{} = \sqrt{\BsT:\BsT} = \sqrt{\tfrac{2}{3}\,q^2\,\BnT:\BnT} = 
     \sqrt{\tfrac{2}{3}\,q^2\,\frac{\BeT^e:\BeT^e}{\Norm{\BeT^e}{}^2}}
     = \sqrt{\tfrac{2}{3}\,q^2} = \sqrt{\tfrac{2}{3}}\,q \,.
\]
So we can write
\Beq
   \Partial{f}{\Bsig} = \frac{2p - p_c}{3}\,\Bone + \sqrt{\tfrac{3}{2}}\,\frac{2q}{M^2}\,\BnT \,.
\Eeq
Using the above relation we have
\[
   \Partial{f}{p} = \tfrac{1}{3}\,\Tr\left[\Partial{f}{\Bsig}\right] = 2p-p_c 
   \quad \Tand \quad
   \Partial{f}{\BsT}  = \Partial{f}{\Bsig} - \Partial{f}{p}\Bone
                      = \sqrt{\tfrac{3}{2}}\,\frac{2q}{M^2}\,\BnT \,.
\]
The strain updates can now be written as
\[
  \Bal
  (\Ve_v^e)_{n+1} & =  (\Ve_v^e)_\Trial - \Delta\gamma\,[2p_{n+1} - (p_c)_{n+1}] \\
  \BeT^e_{n+1} & = \BeT^e_\Trial - \sqrt{\tfrac{3}{2}}\,\Delta\gamma\,\left(\frac{2q_{n+1}}{M^2_{n+1}}\right)
           \BnT_{n+1}  \\
  (\Ve_s^e)_{n+1} & = 
    \sqrt{\tfrac{2}{3}}\,\sqrt{\BeT^e_\Trial:\BeT^e_\Trial 
     - \sqrt{6}\,(\Delta\gamma)^2\left(\frac{2q_{n+1}}{M^2_{n+1}}\right)\BnT_{n+1}:\BeT^e_\Trial
     + \tfrac{3}{2}\,(\Delta\gamma)^4\,\left(\frac{2q_{n+1}}{M^2_{n+1}}\right)^2 } \,.
  \Eal
\]
From the second equation above,
\[
  \BnT_{n+1}:\BeT^e_\Trial = 
   \BnT_{n+1}:\BeT^e_{n+1} + \sqrt{\tfrac{3}{2}}\,\Delta\gamma\,\left(\frac{2q_{n+1}}{M^2_{n+1}}\right)
           \BnT_{n+1}:\BnT_{n+1} = 
   \frac{\BeT^e_{n+1}:\BeT^e_{n+1}}{\Norm{\BeT^e_{n+1}}{}} + 
      \sqrt{\tfrac{3}{2}}\,\Delta\gamma\,\left(\frac{2q_{n+1}}{M^2_{n+1}}\right) =
   \Norm{\BeT^e_{n+1}}{} + 
      \sqrt{\tfrac{3}{2}}\,\Delta\gamma\,\left(\frac{2q_{n+1}}{M^2_{n+1}}\right) \,.
\]
Also notice that
\[
  \BeT^e_\Trial:\BeT^e_\Trial = \BeT^e_{n+1}:\BeT^e_{n+1} + 2\,\sqrt{\tfrac{3}{2}}\,\Delta\gamma\,
     \left(\frac{2q_{n+1}}{M^2_{n+1}}\right)\BeT^e_{n+1}:\BnT_{n+1} +
     \left[\sqrt{\tfrac{3}{2}}\,\Delta\gamma\,\left(\frac{2q_{n+1}}{M^2_{n+1}}\right)\right]^2
\]
or,
\[
  \Norm{\BeT^e_\Trial}{}^2 = \left[\Norm{\BeT^e_{n+1}}{} + \sqrt{\tfrac{3}{2}}\,\Delta\gamma\,\left(\frac{2q_{n+1}}{M^2_{n+1}}\right)\right]^2 \,.
\]
Therefore,
\[
  \BnT_{n+1}:\BeT^e_\Trial = \Norm{\BeT^e_\Trial}{} 
\]
and we have
\[
  (\Ve_s^e)_{n+1} = 
    \sqrt{\tfrac{2}{3}}\,\sqrt{\Norm{\BeT^e_\Trial}{}^2
     - \sqrt{6}\,(\Delta\gamma)^2\left(\frac{2q_{n+1}}{M^2_{n+1}}\right)\Norm{\BeT^e_\Trial}{}
     + \tfrac{3}{2}\,(\Delta\gamma)^4\,\left(\frac{2q_{n+1}}{M^2_{n+1}}\right)^2 } 
    = \sqrt{\tfrac{2}{3}}\,\Norm{\BeT^e_\Trial}{} - \Delta\gamma\,\left(\frac{2q_{n+1}}{M^2_{n+1}}\right) \,.
\]
The elastic strain can therefore be updated using
\[
  \Bal 
    (\Ve_v^e)_{n+1} & =  (\Ve_v^e)_\Trial - \Delta\gamma\,[2p_{n+1} - (p_c)_{n+1}] \\
    (\Ve_s^e)_{n+1} & =  (\Ve_s^e)_\Trial - \Delta\gamma\,\left(\frac{2q_{n+1}}{M^2_{n+1}}\right) \,.
  \Eal 
\]
The consistency condition is needed to close the above equations
\[
   f = \left(\frac{q_{n+1}}{M}\right)^2 + p_{n+1}[p_{n+1}-(p_c)_{n+1}] = 0  \,.
\]
The unknowns are $(\Ve_v^e)_{n+1}$, $(\Ve_s^e)_{n+1}$ and $\Delta\gamma$.  Note that we can express
the three equations as
\Beq
  \Bal 
    (\Ve_v^e)_{n+1} & =  (\Ve_v^e)_\Trial - \Delta\gamma\,\left[\Partial{f}{p}\right]_{n+1} \\
    (\Ve_s^e)_{n+1} & =  (\Ve_s^e)_\Trial - \Delta\gamma\,\left[\Partial{f}{q}\right]_{n+1} \\
    f_{n+1} & = 0 \,.
  \Eal 
\Eeq

\subsection{Newton iterations}
The three nonlinear equations in the three unknowns can be solved using Newton iterations
for smooth yield functions.  Let us define the residual as
\[
   \Mr(\Mx) = \begin{bmatrix} 
    (\Ve_v^e)_{n+1} -  (\Ve_v^e)_\Trial + \Delta\gamma\,\left[\Partial{f}{p}\right]_{n+1} \\
    (\Ve_s^e)_{n+1} -  (\Ve_s^e)_\Trial + \Delta\gamma\,\left[\Partial{f}{q}\right]_{n+1} \\
    f_{n+1} \end{bmatrix} =: \begin{bmatrix} r_1 \\ r_2 \\ r_3 \end{bmatrix}
   \quad \text{where} \quad
   \Mx = \begin{bmatrix} (\Ve_v^e)_{n+1} \\ (\Ve_s^e)_{n+1} \\ f_{n+1} \end{bmatrix} 
        =: \begin{bmatrix} x_1 \\ x_2 \\ x_3 \end{bmatrix} \,.
\]
The Newton root finding algorithm is :
\begin{algorithm}
  \begin{algorithmic}
    \REQUIRE $\Mx^0$
    \STATE $k \leftarrow 0$
    \WHILE {$\Mr(\Mx^k) \ne 0$}
      \STATE $\Mx^{k+1} \Leftarrow \Mx^k - \left[\left(\Partial{\Mr}{\Mx}\right)^{-1}\right]_{\Mx^k}\cdot
              \Mr(\Mx^k)$
      \STATE $k \leftarrow k+1$
    \ENDWHILE
  \end{algorithmic}
\end{algorithm}

To code the algorithm we have to find the derivatives of the residual with respect to the primary variables.
Let's do the terms one by one.  For the first row,
\[
  \Bal
  \Partial{r_1}{x_1} & = \Partial{}{\Ve_v^e}\left[\Ve_v^e -  (\Ve_v^e)_\Trial + \Delta\gamma\,(2p-p_c)\right] 
     = 1 + \Delta\gamma\left(2\Partial{p}{\Ve_v^e} - \Partial{p_c}{\Ve_v^e}\right) \\
  \Partial{r_1}{x_2} & = \Partial{}{\Ve_s^e}\left[\Ve_v^e -  (\Ve_v^e)_\Trial + \Delta\gamma\,(2p-p_c)\right] 
     = 2\Delta\gamma\,\Partial{p}{\Ve_s^e}\\
  \Partial{r_1}{x_3} & = \Partial{}{\Delta\gamma}\left[\Ve_v^e -  (\Ve_v^e)_\Trial + \Delta\gamma\,(2p-p_c)\right]
     = 2p - p_c = \Partial{f}{p}
  \Eal
\]
where
\[
  \Bal
   \Partial{p}{\Ve_v^e} & = -\frac{p_0\,\beta}{\kappatilde}\,\exp\left[-\frac{\Ve_v^e - \Ve_{v0}^e}{\kappatilde}\right] = \frac{p}{\kappatilde} \quad, \quad
   \Partial{p_c}{\Ve_v^e} = \frac{(p_c)_n}{\kappatilde-\lambdatilde}\,\exp\left[\frac{\Ve_v^e - (\Ve_{v}^e)_\Trial}{\kappatilde-\lambdatilde}\right] \quad \Tand \\
   \Partial{p}{\Ve_s^e} & = \frac{3\,p_0\,\alpha\,\Ve_s^e}{\kappatilde}\,\exp\left[-\frac{\Ve_v^e - \Ve_{v0}^e}{\kappatilde}\right] \,.
  \Eal
\]
For the second row,
\[
  \Bal
  \Partial{r_2}{x_1} & = \Partial{}{\Ve_v^e}\left[\Ve_s^e -  (\Ve_s^e)_\Trial + \Delta\gamma\,\frac{2q}{M^2}\right] 
     = \frac{2\Delta\gamma}{M^2}\,\Partial{q}{\Ve_v^e} \\
  \Partial{r_2}{x_2} & = \Partial{}{\Ve_s^e}\left[\Ve_s^e -  (\Ve_s^e)_\Trial + \Delta\gamma\,\frac{2q}{M^2}\right] 
     = 1 + \frac{2\Delta\gamma}{M^2}\,\Partial{q}{\Ve_s^e}\\
  \Partial{r_2}{x_3} & = \Partial{}{\Delta\gamma}\left[\Ve_s^e -  (\Ve_s^e)_\Trial + \Delta\gamma\,\frac{2q}{M^2}\right]
     = \frac{2q}{M^2} = \Partial{f}{q}
  \Eal
\]
where
\[
  \Partial{q}{\Ve_v^e} = -\frac{3p_0\,\alpha\,\Ve_s^e}{\kappatilde}\,\exp\left[-\frac{\Ve_v^e - \Ve_{v0}^e}{\kappatilde}\right] = \Partial{p}{\Ve_s^e}
  \quad \Tand \quad 
  \Partial{q}{\Ve_s^e} = 3\mu_0 + 3p_0\,\alpha\,\exp\left[-\frac{\Ve_v^e - \Ve_{v0}^e}{\kappatilde}\right] = 3\mu \,.
\]
For the third row, 
\[
  \Bal
  \Partial{r_3}{x_1} & = \Partial{}{\Ve_v^e}\left[\frac{q^2}{M^2} + p\,(p - p_c)\right]
     = \frac{2q}{M^2}\,\Partial{q}{\Ve_v^e} + (2p - p_c)\,\Partial{p}{\Ve_v^e} - p\,\Partial{p_c}{\Ve_v^e} 
     = \Partial{f}{q}\,\Partial{q}{\Ve_v^e} + \Partial{f}{p}\,\Partial{p}{\Ve_v^e} - p\,\Partial{p_c}{\Ve_v^e} \\
  \Partial{r_3}{x_2} & = \Partial{}{\Ve_s^e}\left[\frac{q^2}{M^2} + p\,(p - p_c)\right]
     = \frac{2q}{M^2}\,\Partial{q}{\Ve_s^e} + (2p - p_c)\,\Partial{p}{\Ve_s^e} 
     = \Partial{f}{q}\,\Partial{q}{\Ve_s^e} + \Partial{f}{p}\,\Partial{p}{\Ve_s^e} \\
  \Partial{r_3}{x_3} & = \Partial{}{\Delta\gamma}\left[\frac{q^2}{M^2} + p\,(p - p_c)\right]
     =  0 \,.
  \Eal
\]
We have to invert a matrix in the Newton iteration process.  Let us see whether we can make this 
quicker to do.  The Jacobian matrix has the form
\[
  \Partial{\Mr}{\Mx} = \begin{bmatrix}\Partial{r_1}{x_1} & \Partial{r_1}{x_2} & \Partial{r_1}{x_3} \\
     \Partial{r_2}{x_1} & \Partial{r_2}{x_2} & \Partial{r_2}{x_3} \\
     \Partial{r_3}{x_1} & \Partial{r_3}{x_2} & \Partial{r_3}{x_3} \end{bmatrix} 
     = \begin{bmatrix} \MAmat & \MBmat \\ \MCmat & 0 \end{bmatrix}
\]
where
\[
  \MAmat = \begin{bmatrix}\Partial{r_1}{x_1} & \Partial{r_1}{x_2} \\
                       \Partial{r_2}{x_1} & \Partial{r_2}{x_2} \end{bmatrix} \,,\quad
  \MBmat = \begin{bmatrix}\Partial{r_1}{x_3} \\ \Partial{r_2}{x_3} \end{bmatrix} \,,\quad \Tand \quad
  \MCmat = \begin{bmatrix}\Partial{r_3}{x_1} & \Partial{r_3}{x_2} \end{bmatrix} \,.
\]
We can also break up the $\Mx$ and $\Mr$ matrices:
\[
  \Delta\Mx = \Mx^{k+1}-\Mx^k = \begin{bmatrix} \Delta\Mx^{vs} \\ \Delta x_3 \end{bmatrix} \,, \quad
  \Mr = \begin{bmatrix} \Mr^{vs} \\ r_3 \end{bmatrix}
  \quad \text{where} \quad \Mr^{vs} = \begin{bmatrix} r_1 \\ r_2 \end{bmatrix}  
   \quad \Tand \quad \Delta\Mx^{vs} = \begin{bmatrix}\Delta x_1 \\\Delta x_2 \end{bmatrix}  \,.
\]
Then
\[
  \begin{bmatrix} \Delta\Mx^{vs} \\ \Delta x_3 \end{bmatrix} 
   = - \begin{bmatrix} \MAmat & \MBmat \\ \MCmat & 0 \end{bmatrix}^{-1} 
                    \begin{bmatrix} \Mr^{vs} \\ r_3 \end{bmatrix} 
  \quad \implies \quad 
   \begin{bmatrix} \MAmat & \MBmat \\ \MCmat & 0 \end{bmatrix} 
  \begin{bmatrix} \Delta\Mx^{vs} \\ \Delta x_3 \end{bmatrix}  = 
                    -\begin{bmatrix} \Mr^{vs} \\ r_3 \end{bmatrix} 
\]
or
\[
   \MAmat\,\Delta\Mx^{vs} + \MBmat\,\Delta x_3 = -\Mr^{vs} \quad \Tand \quad
   \MCmat\,\Delta\Mx^{vs} = -r_3 \,.
\]
From the first equation above,
\[
  \Delta\Mx^{vs} = -\MAmat^{-1}\,\Mr^{vs} - \MAmat^{-1}\,\MBmat\,\Delta x_3 \,.
\]
Plugging in the second equation gives
\[
   r_3 = \MCmat\,\MAmat^{-1}\,\Mr^{vs} + \MCmat\,\MAmat^{-1}\,\MBmat\,\Delta x_3 \,.
\]
Rearranging,
\[
  \Delta x_3 = x_3^{k+1} - x_3^k = \frac{-\MCmat\,\MAmat^{-1}\,\Mr^{vs} + r_3}{\MCmat\,\MAmat^{-1}\,\MBmat} \,.
\]
Using the above result,
\[
  \Delta\Mx^{vs} = -\MAmat^{-1}\,\Mr^{vs} - \MAmat^{-1}\,\MBmat\,\left(\frac{-\MCmat\,\MAmat^{-1}\,\Mr^{vs} + r_3}{\MCmat\,\MAmat^{-1}\,\MBmat}\right) \,.
\]
We therefore have to invert only a $2 \times 2 $ matrix.

\subsection{Tangent calculation: elastic}
We want to find the derivative of the stress with respect to the strain:
\Beq
   \Partial{\Bsig}{\Beps} = \Bone\otimes\Partial{p}{\Beps} + 
      \sqrt{\tfrac{2}{3}}\,\BnT\otimes\Partial{q}{\Beps} + 
      \sqrt{\tfrac{2}{3}}\,q\,\Partial{\BnT}{\Beps}  \,.
\Eeq
For the first term above,
\[
   \Partial{p}{\Beps} = p_0\,\exp\left[-\frac{\Ve^e_v - \Ve^e_{v0}}{\kappatilde}\right]\Partial{\beta}{\Beps}
      - p_0\,\frac{\beta}{\kappatilde}\,
        \exp\left[-\frac{\Ve^e_v - \Ve^e_{v0}}{\kappatilde}\right]\Partial{\Ve^e_v}{\Beps} 
     = p_0\,\exp\left[-\frac{\Ve^e_v - \Ve^e_{v0}}{\kappatilde}\right]\left(\Partial{\beta}{\Beps} -
            \frac{\beta}{\kappatilde}\,\Partial{\Ve^e_v}{\Beps} \right) \,.
\]
Now,
\[
   \Partial{\beta}{\Beps} = \frac{3\alpha}{\kappatilde}\,\Ve^e_s\,\Partial{\Ve^e_s}{\Beps} \,.
\]
Therefore, 
\[
  \Partial{p}{\Beps} = \frac{p_0}{\kappatilde}\,
      \exp\left[-\frac{\Ve^e_v - \Ve^e_{v0}}{\kappatilde}\right]\left(3\alpha\,\Ve^e_s\Partial{\Ve^e_s}{\Beps} -
            \beta\,\Partial{\Ve^e_v}{\Beps} \right) \,.
\]
We now have to figure out the other derivatives in the above expression.  First,
\[
  \Partial{\Ve^e_s}{\Beps} = \sqrt{\tfrac{2}{3}}\,\frac{1}{\sqrt{\BeT^e:\BeT^e}}\,\Partial{\BeT^e}{\Beps}:\BeT^e =
     \sqrt{\tfrac{2}{3}}\,\frac{1}{\Norm{\BeT^e}{}}\,
     \left(\Partial{\Beps^e}{\Beps} - \tfrac{1}{3}\Bone\otimes\Partial{\Ve^e_v}{\Beps} \right):\BeT^e\,.
\]
For the special situation where all the strain is elastic, $\Beps = \Beps^e$, and (see Wikipedia 
article on tensor derivatives)
\[
  \Partial{\Beps^e}{\Beps} = \Partial{\Beps}{\Beps} = \SfI^{(s)} \quad \Tand \quad
  \Partial{\Ve^e_v}{\Beps} = \Partial{\Ve_v}{\Beps} = \Bone \,.
\]
That gives us
\[
  \Partial{\Ve^e_s}{\Beps} = \sqrt{\tfrac{2}{3}}\,\frac{1}{\Norm{\BeT^e}{}}\,
     \left(\SfI^{(s)} - \tfrac{1}{3}\Bone\otimes\Bone \right):\BeT^e
   = \sqrt{\tfrac{2}{3}}\,\frac{1}{\Norm{\BeT^e}{}}\,
     \left[\BeT^e - \tfrac{1}{3}\Tr(\BeT^e)\Bone\right] \,.
\]
But $\Tr(\BeT^e)=0$ because this is the deviatoric part of the strain and we have
\[
  \boxed{
  \Partial{\Ve^e_s}{\Beps} = \sqrt{\tfrac{2}{3}}\,\frac{\BeT^e}{\Norm{\BeT^e}{}} = \sqrt{\tfrac{2}{3}}\,\BnT 
  }
  \quad \Tand \quad
  \boxed{
  \Partial{\Ve^e_v}{\Beps} = \Bone \,.
  }
\]
Using these, we get
\Beq
  \Partial{p}{\Beps} = \frac{p_0}{\kappatilde}\,
      \exp\left[-\frac{\Ve^e_v - \Ve^e_{v0}}{\kappatilde}\right]\left(\sqrt{6}\,\alpha\,\Ve^e_s\,\BnT -
            \beta\,\Bone \right) \,.
\Eeq
The derivative of $q$ with respect to $\Beps$ can be calculated in a similar way, i.e.,
\[
  \Partial{q}{\Beps} = 3\mu\,\Partial{\Ve^e_s}{\Beps} + 3\Ve^e_s\,\Partial{\mu}{\Beps}
   = 3\mu\,\Partial{\Ve^e_s}{\Beps} - 3\frac{p_0}{\kappatilde}\,\alpha\,\Ve^e_s\,
      \exp\left[-\frac{\Ve^e_v - \Ve^e_{v0}}{\kappatilde}\right]\,\Partial{\Ve^e_v}{\Beps} \,.
\]
Using the expressions in the boxes above, 
\Beq
  \Partial{q}{\Beps} = \sqrt{6}\,\mu\,\BnT - 3\frac{p_0}{\kappatilde}\,
      \exp\left[-\frac{\Ve^e_v - \Ve^e_{v0}}{\kappatilde}\right]\,\alpha\,\Ve^e_s\,\Bone \,.
\Eeq
Also,
\[
   \Partial{\BnT}{\Beps} = \sqrt{\tfrac{2}{3}}\,\left[\frac{1}{\Ve^e_s}\,\Partial{\BeT^e}{\Beps}
     - \frac{1}{(\Ve^e_s)^2}\,\BeT^e\otimes\Partial{\Ve^e_s}{\Beps}\right] \,.
\]
Using the previously derived expression, we have
\[
   \Partial{\BnT}{\Beps} = \sqrt{\tfrac{2}{3}}\,\frac{1}{\Ve^e_s}\,\left[
        \SfI^{(s)} - \tfrac{1}{3}\,\Bone\otimes\Bone
     - \sqrt{\tfrac{2}{3}}\,\frac{1}{\Ve^e_s}\,\frac{\BeT^e\otimes\BeT^e}{\Norm{\BeT^e}{}}\right] 
\]
or
\Beq
   \Partial{\BnT}{\Beps} = \sqrt{\tfrac{2}{3}}\,\frac{1}{\Ve^e_s}\,\left[
        \SfI^{(s)} - \tfrac{1}{3}\,\Bone\otimes\Bone - \BnT\otimes\BnT\right] \,.
\Eeq
Plugging the expressions for these derivatives in the original equation, we get
\[
  \Bal
   \Partial{\Bsig}{\Beps} & = \frac{p_0}{\kappatilde}\,
      \exp\left[-\frac{\Ve^e_v - \Ve^e_{v0}}{\kappatilde}\right]
      \left(\sqrt{6}\,\alpha\,\Ve^e_s\,\Bone\otimes\BnT - \beta\,\Bone\otimes\Bone \right) + 
      2\mu\,\BnT\otimes\BnT - \sqrt{6}\frac{p_0}{\kappatilde}\,
      \exp\left[-\frac{\Ve^e_v - \Ve^e_{v0}}{\kappatilde}\right]\,\alpha\,\Ve^e_s\,\BnT\otimes\Bone +\\
      & \qquad \qquad \tfrac{2}{3}\,\frac{q}{\Ve^e_s}\,\left[
        \SfI^{(s)} - \tfrac{1}{3}\,\Bone\otimes\Bone - \BnT\otimes\BnT\right] \,.
  \Eal
\]
Reorganizing,
\Beq
  \boxed{
  \Bal
   \Partial{\Bsig}{\Beps} & = \frac{\sqrt{6}\,p_0\,\alpha\,\Ve^e_s}{\kappatilde}\,
      \exp\left[-\frac{\Ve^e_v - \Ve^e_{v0}}{\kappatilde}\right](\Bone\otimes\Bn + \Bn\otimes\Bone) - 
      \left(\frac{p_0\beta}{\kappatilde}\, \exp\left[-\frac{\Ve^e_v - \Ve^e_{v0}}{\kappatilde}\right]
       +\tfrac{2}{9}\,\frac{q}{\Ve_s^e}\right) \Bone\otimes\Bone + \\
     & \qquad \qquad 2\left(\mu - \tfrac{1}{3}\,\frac{q}{\Ve^e_s}\right)\,\BnT\otimes\BnT 
           + \tfrac{2}{3}\,\frac{q}{\Ve^e_s}\,\SfI^{(s)}\,.
  \Eal
  }
\Eeq

\subsection{Tangent calculation: elastic-plastic}
From the previous section recall that 
\[
   \Partial{\Bsig}{\Beps} = \Bone\otimes\Partial{p}{\Beps} + 
      \sqrt{\tfrac{2}{3}}\,\BnT\otimes\Partial{q}{\Beps} + 
      \sqrt{\tfrac{2}{3}}\,q\,\Partial{\BnT}{\Beps}  
\]
where
\[
  \Bal
  \Partial{p}{\Beps} & = \frac{p_0}{\kappatilde}\,
      \exp\left[-\frac{\Ve^e_v - \Ve^e_{v0}}{\kappatilde}\right]\left(3\alpha\,\Ve^e_s\Partial{\Ve^e_s}{\Beps} -
            \beta\,\Partial{\Ve^e_v}{\Beps} \right) \,,\qquad
  \Partial{q}{\Beps}  = 3\mu\,\Partial{\Ve^e_s}{\Beps} - 3\frac{p_0}{\kappatilde}\,\alpha\,\Ve^e_s \,
      \exp\left[-\frac{\Ve^e_v - \Ve^e_{v0}}{\kappatilde}\right]\,\Partial{\Ve^e_v}{\Beps} \quad \Tand \\
   \Partial{\BnT}{\Beps} & = \sqrt{\tfrac{2}{3}}\,\left[\frac{1}{\Ve^e_s}\,\Partial{\BeT^e}{\Beps}
     - \frac{1}{(\Ve^e_s)^2}\,\BeT^e\otimes\Partial{\Ve^e_s}{\Beps}\right] \,.
  \Eal
\]
The total strain is equal to the elastic strain for the purely elastic case and the tangent is relatively
straightforward to calculate.  For the elastic-plastic case we have
\[
   \Beps^e_{n+1} = \Beps^e_\Trial - \Delta\gamma\left[\Partial{f}{\Bsig}\right]_{n+1} \,.
\]
Dropping the subscript $n+1$ for convenience, we have
\[
   \Partial{\Beps^e}{\Beps} = \Partial{\Beps^e_\Trial}{\Beps} 
     - \Partial{f}{\Bsig}\otimes\Partial{\Delta\gamma}{\Beps}
     - \Delta\gamma\,\Partial{}{\Beps}\left[\Partial{f}{\Bsig}\right]
     = \SfI^{(s)}
     - \left[\frac{2p-p_c}{3}\,\Bone + \sqrt{\tfrac{3}{2}}\,\frac{2q}{M^2}\,\BnT\right]
       \otimes\Partial{\Delta\gamma}{\Beps}
     - \Delta\gamma\,\Partial{}{\Beps}
     \left[\frac{2p-p_c}{3}\,\Bone + \sqrt{\tfrac{3}{2}}\,\frac{2q}{M^2}\,\BnT\right] \,.
\]


%\begin{figure}[htb!]
%  \centering
%  %\includegraphics[width=0.4\linewidth]{./FIGS/SimpleBurnWithoutPressCC.png}
%  %\includegraphics[width=0.4\linewidth]{./FIGS/SimpleBurnWithPressCC.png}
%  \caption{Simulation of an explosion in a half space overlaid by a thin layer with a
%           simple burn model and compressible neo-Hookean behavior.  There is no apparent
%           fast moving shock wave in the material because the speed of sound in the half space
%           is low.  Movies of this simulation can be found in the Dropbox folder
%           SoilWithLayerSimpleBurn2D.}
%  \label{fig:SimpleBurn}
%\end{figure}


\end{document}
