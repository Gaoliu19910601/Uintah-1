\documentclass[fleqn,reqno]{amsart}      
\usepackage{amsbsy}
\allowdisplaybreaks
\newcommand{\pdd}[2]{\frac{\partial{#1}}{\partial{#2}}}
\newcommand{\spdd}[3]{\frac{\partial^{2}{#1}}{\partial{#2}\partial{#3}}}
\newcommand{\sd}[2]{\frac{\partial^{2}{#1}}{\partial{#2}^2}}
\newcommand{\av}[1]{\overline{#1}}
\newcommand{\ldd}[2]{\frac{d{#1}}{d{#2}}}
\newcommand{\inlineldd}[2]{d{#1}/d{#2}}
%remove the following command to get rid of double space.
\linespread{1.6}
\title{Fire in single material ICE.}
\author{Stanislav G. Borodai}
\begin{document}        
\maketitle
\section{Introduction.}
The following discussion represents a pathway as to how to introduce
combustion in ICE code. It is only an initial proposal and should be treated as
such. Only single material ICE combustion implementation will be discussed,
the generalization to multi-material would be derived after satisfactory progress is
obtained in single material combustion modeling in ICE code. Also, the initial task
would be only to incorporate mixing and reaction models in ICE, other necessary
models to perform LES pool fire simulation in ICE such as turbulence models,
radiation models, etc. would be discussed and implemented upon satisfactory progress obtained
on this initial task. This means that the only simulation possible to achieve
after completion of this step is laminar compressible adiabatic fire.
The ability of such simulation to handle long time scale (on the order of 10 seconds)
fires has then to be verified.
However, even
before this simulation, simulation of laminar air jet and helium laminar jet in air need to
be performed. First one is to verify the boundary conditions available in ICE,
second is to verify ability of ICE to handle variable density flows upon
initial implementation of cold flow mixing models (no reaction). Along with
these simulations, standard ICE simulations like pressure wave propagation
need to be performed upon implementation of mixing and reaction models to
verify that new formulation preserves the ability of ICE to handle standard
compressible flow cases.

\section{General considerations.}
\label{s1}
The main idea behind introducing combustion in ICE comes from examining
a general form of a subgrid scale equation of state used in combustion:
\begin{align}
\label{e1}
\rho=\rho(p,f,h,...),
\end{align}
where $\rho$ is density, $p$ is pressure, $f$ is mixture fraction,
$h$ is enthalpy and $...$ are other combustion related variables like
reaction progress variable, omitted here without loss of generality.
From (\ref{e1}), convective or Lagrangian time derivative of $\rho$
would be
\begin{align}
\label{e2}
\ldd{\rho}{t}=\left.\pdd{\rho}{p}\right|_{f,h}\ldd{p}{t}+
\left.\pdd{\rho}{f}\right|_{p,h}\ldd{f}{t}+
\left.\pdd{\rho}{h}\right|_{p,f}\ldd{h}{t},
\end{align}
where subscripts indicate that the derivatives are computed
with these variables being constant.
From (\ref{e2}) and continuity equation, one obtains divergence
constraint for the velocity $\boldsymbol{u}$:
\begin{align}
\label{e3}
\pdd{u_i}{x_i}=-\frac{1}{\rho}\left.\pdd{\rho}{p}\right|_{f,h}\ldd{p}{t}-
\frac{1}{\rho}\left.\pdd{\rho}{f}\right|_{p,h}\ldd{f}{t}-
\frac{1}{\rho}\left.\pdd{\rho}{h}\right|_{p,f}\ldd{h}{t}.
\end{align}
Lagrangian time derivatives of independent variables in~(\ref{e3}), except
pressure,
have to be computed explicitly each time step, and partial derivatives in~(\ref{e3})
will be computed from tabulated sungrid scale combustion models. 
The divergence constraint~(\ref{e3}) will be used in the modified ICE
pressure projection in section~\ref{s3} to include combustion effects.


\section{Governing equations}
\label{s2}
Let us examine the subset of governing equations involved. Only the equations
pertaining to the combustion will be presented, along with the equation for
fluxing velocity necessary to explain the method. The rest of ICE algorithm
can be used unchanged. The equations are: 
\begin{align}
\label{e4}
\ldd{\rho f}{t}=\pdd{}{x_j}\left(D_f \pdd{f}{x_j}\right),
\end{align}
\begin{align}
\label{e5}
\ldd{\rho h}{t}=\pdd{}{x_j}\left(D_h \pdd{h}{x_j}\right)+\ldd{p}{t}+Q,
\end{align}
\begin{align}
\label{e6}
\ldd{\rho u^f_i}{t}=-\pdd{p}{x_i}+\pdd{\mu s_{ij}}{x_j}+(\rho_{ref}-\rho)g_i,
\end{align}
where $D_f$ and $D_h$ are the resolved scale diffusion coefficients for mixture fraction and enthalpy respectively,
$Q$ is heat loss or gain due to radiation, $\boldsymbol{u}^f$ is fluxing velocity, $\mu$ is viscosity,
$\rho_{ref}$ is reference density, $\boldsymbol{g}$ is gravity vector and $s_{ij}$ is twice the deviatoric
part of strain-rate tensor:
\begin{align}
\label{e7}
s_{ij}=\pdd{u_i}{x_j}+\pdd{u_j}{x_i}-\frac{2}{3}\delta_{ij}\pdd{u_i}{x_i}.
\end{align}
Note that, as written, the resolved scale diffusion coefficients and viscosity combine molecular effects, and LES model effects.
The equation of state~(\ref{e1}) from section~\ref{s1} finishes the set of equations to be discussed.
Also note that in the current formulation
the energy equation is presented in enthalpy form. The advantage of this form is that
(when there is no radiation) the absolute enthalpy is a conserved quantity in combustion process,
no explicit reaction rate source terms, which are often hard to compute,
are necessary in the energy equation. This set of equations, except that for the
fluxing velocity, is taken from the current Arches approach. However, in this first approach
it is also assumed that Diffusion coefficients for mixture fraction and enthalpy
are the same and proportional to viscosity, with a constant proportionality
coefficient.

\section{Modified single material ICE algorithm.}
\label{s3}
In this section the discrete ICE algorithm, modified to include combustion effects,
will be discussed. Since the current goal of ICE is to run in implicit
mode for pressure to overcome acoustic CFL time step limit, only the modification
to implicit in pressure ICE will be discussed. Explicit ICE extension can be
obtained in the same manner, if necessary. Also, since everything but pressure
is computed explicitly, the operator splitting errors are unavoidable. Thus,
the implicit iteration encompassing this whole part of algorithm may be
required to remove these errors.

\begin{enumerate}
\item{Assuming that all the information for some time $N$ is available, the first step would
be to compute $f^{N+1}$ and $h^{*,N+1}$ and their Lagrangian derivatives from
equations (\ref{e4})-(\ref{e5}). The Lagrangian time derivative of pressure
on the right hand side of (\ref{e5}) is not available, so it would be assumed zero for this step.
That is why enthalpy has star superscript.
Other splitting errors in this step would be produced form division of $(\inlineldd{\rho f}{t})^{N+1}$, 
$(\inlineldd{\rho h}{t})^{N+1}$, $(\rho f)^{N+1}$, $(\rho h)^{N+1}$ by $\rho^N$ to
obtain the values in question. If no radiation model is present, then $Q$ term in
enthalpy equation (\ref{e5}) becomes zero, and enthalpy equation for this step becomes exactly
the same as mixture fraction equation (\ref{e4}). This means that flame is adiabatic with
no pressure effects. In this case adiabatic enthalpy can be computed from simple formula:
\begin{align}
\label{e8}
h^{*,N+1}=h^{*,N+1}_a=f*h_{fuel}+(1-f)*h_{air},
\end{align}
given heats of formation $h_{fuel}$ and $h_{air}$ for fuel and air respectively, so that
explicit transport of enthalpy by equation (\ref{e5}) is not necessary.}

\item{Second step is to get predicted value of $\rho^{*,N+1}$ given $f^{N+1}$,$h^{*,N+1}$, and old pressure $p^N$.
This step, in general, would be a subgrid scale chemistry table lookup.
Partial derivatives for
equation (\ref{e3}) should be obtained from table at this step as well. 
In adiabatic case equation (\ref{e3}) does not depend on enthalpy. Note that one
more splitting error has been introduced, since new value of pressure is not yet
available.}

\item{Third step would be computation of fluxing velocity and pressure projection. 
One can define pressure correction as:
\begin{align}
\label{e9}
p^c=p^{N+1}-p^{N}.
\end{align}
Then, Lagrangian value
of fluxing velocity $\boldsymbol{u}^{f*,N+1}$ is defined as (also using (\ref{e9}) in (\ref{e6})):
\begin{align}
\label{e10}
&u^{f*,N+1}_i=\nonumber\\
&\frac{1}{\rho^{*,N+1}}\left(
u^{f,N}+\Delta t \left(-\pdd{p^N}{x_i}+\pdd{}{x_j}\left(\mu^N \pdd{s^N_{ij}}{x_j}\right)+
(\rho_{ref}-\rho^N)g_i\right)\right).
\end{align}
The velocity $\boldsymbol{u}^{f*,N+1}$ includes all the right hand side of (\ref{e6}) but the gradient of pressure correction term.
Again, one more splitting error has been introduced, since $\rho^{N+1}$ is not available yet.
The actual fluxing velocity $\boldsymbol{u}^{f,N+1}$ can then be obtained from the equation:
\begin{align}
\label{e11}
u^{f,N+1}_i=u^{f*,N+1}_i-\frac{\Delta t}{\rho^{*,N+1}}\pdd{p^c}{x_i},
\end{align}
after pressure correction $p^c$ has been solved for.

To derive the equation for pressure, one would apply divergence operator to the both sides of (\ref{e11})
\begin{align}
\label{e12}
\pdd{u^{f,N+1}_i}{x_i}=\pdd{u^{f*,N+1}_i}{x_i}-\pdd{}{x_i}\left(\frac{\Delta t}{\rho^{*,N+1}}\pdd{p^c}{x_i}\right),
\end{align}
and assuming that divergence of fluxing velocity at time $N+1$ is given by constraint (\ref{e3}),
using (~\ref{e9}) in (\ref{e3}), one would get the equation for the pressure correction $p^c$:
\begin{align}
\label{e13}
&\left(\pdd{}{x_i}\left(\frac{\Delta t}{\rho^{*,N+1}}\pdd{}{x_i}\right) 
-\frac{1}{\rho^{*,N+1} \Delta t}\left.\pdd{\rho}{p}\right|_{f,h}^{N+1}\right)p^c=\nonumber\\
&\pdd{u^{f*,N+1}_i}{x_i} + 
\frac{1}{\rho^{*,N+1}}\left.\pdd{\rho}{f}\right|_{p,h}^{N+1}{\ldd{f}{t}}^{N+1}+
\frac{1}{\rho^{*,N+1}}\left.\pdd{\rho}{h}\right|_{p,f}^{*,N+1}{\ldd{h}{t}}^{*,N+1},
\end{align}
where the first term on the right hand side of (\ref{e13}) can be computed taking divergence of (\ref{e10}).
Again, all the variables with star subscript, except for $\boldsymbol{u}^{f,N+1}$, indicate that splitting
error has been introduced in their computation.
Now, pressure correction $p^c$ can be obtained solving (\ref{e13}) with appropriate boundary conditions.
Note that if right hand side of (\ref{e13}) and boundary conditions do not depend on $p^c$, no outer
iteration while solving (\ref{e13}) is required. After pressure correction has been solved for,
fluxing velocity can be obtained from (\ref{e10}).}

\item{The last step would be to update enthalpy with now known pressure effect, if desired:
\begin{align}
\label{e14}
h^{N+1}=h^{*,N+1}+\frac{p^c}{\rho^{*,N+1}\Delta t},
\end{align}
and get updated density $\rho^{N+1}$ by table lookup, but now as a function of $f^{N+1}$, ${h^{N+1}}$, and $p^{N+1}$ now
available from (\ref{e9}).
This step concludes the modification to the ICE algorithm, the rest of it can be done as usual.
Again, please not that with all splitting errors committed, it may be necessary to perform the procedure above
in implicit manner. Also, the Lagrangian pressure time derivative present in equation (\ref{e5}) can be
treated the same way as it was done in (\ref{e13}), but benefit of doing so requires further investigation.}
\end{enumerate}


\end{document} 
