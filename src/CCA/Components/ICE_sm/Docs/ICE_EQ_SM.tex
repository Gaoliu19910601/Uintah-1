\documentclass[fleqn]{article}
\usepackage{amsmath}
\usepackage{setspace}
%\usepackage{algorithmic}
\usepackage{color}
\usepackage{vmargin}
%\setpapersize{USletter}
\setmarginsrb{1.0in}{1.0in}{1.0in}{1.0in}{0pt}{0mm}{0pt}{0mm}
\doublespacing

%..............................
% Alias commands
%..............................    
\newcommand{\SUM}[1]    {\ensuremath{\sum \limits_{{#1}=1}^N }}
\newcommand{\bigS}[1]   {\ensuremath{S_{#1}}  } 
\newcommand{\B}[1]      {\biggr{#1}}            
\newcommand{\U}         {{\vec{U}}}                    
\newcommand{\uo}        {\ensuremath{\vec{U}_o}}      
\newcommand{\rhoM}      {\ensuremath{\rho^{o}}}                 
\newcommand{\delt}      {\ensuremath{\Delta{t}} }                 
\newcommand{\delx}      {\ensuremath{\Delta{x}} } 
\newcommand{\f}         {\ensuremath{f^{\theta}} }
\newcommand{\sv}[1]     {\ensuremath{v^o_{{#1}} }} 
\newcommand{\dpdrho}    {\ensuremath{ \frac{dP}{d\rhoM}} }
\newcommand{\compute}   {\ensuremath{\quad{\text{compute}\rightarrow}\quad}}

 %.....................................................
% This command is necessary so that the subscripts, some of which are
% in capital letters come out
%\DeclareMathSizes{10}{10}{6}{5}
%______________________________________________________________________
\begin{document}
\setlength{\abovedisplayskip}{0.1in}
\setlength{\mathindent}{0.0in}
\doublespacing
\underline{ I C E, explicit pressure, single material  \hspace{ 0.5in}11/7/14}

\begin{enumerate}
%
%__________________________________
\item \underline{Compute Thermodynamic/Transport Properties
$C_v, k, \mu, \gamma$}
%
%__________________________________
\item \underline{Compute the pressure using an equation of state (ideal Gas).}\\
$P_{eq} = (\gamma -1) C_v \rho T$\\
$c = \sqrt{ \frac{dP}{d\rho} + \frac{dP}{de} \frac{P_{eq}}{\rho^2} }$\\
where
$\frac{dP}{d\rho} = (\gamma -1) C_v T$ and 
$\frac{dP}{de} = (\gamma - 1) \rho$ \\
set boundary conditions on $P_{eq}$\\
Compute the compressibility $\kappa = \frac{v^o}{c^2}$
%
%__________________________________
\item \underline{Compute the face-centered velocities}\\
  $ \vec{U}^{*^{f}} 
    = \B{<} \frac{ \rho \U }{\rho} \B{>}^{n^{f}}
    - \frac{ \delt }{< \rhoM >^f} \nabla^{f} P_{eq}
     + \delt\vec{g} $ \\
$    
   =\frac{(\rho \U)_{R} + (\rho \U)_{L}}{\rho_{R} + \rho_{L}}
   -\delt \frac{2.0 (\sv{L} \sv{R})}  {\sv{L} + \sv{R}}  \B{(} \frac{P_{eq_R} - P_{eq_L}}{\Delta x}\B{)}
   + \delt\vec{g} $ \\
set boundary conditions on $\vec{U}^{*^{f}}$
%
%__________________________________
\item \underline{Compute the change in pressure $\Delta P$}  \\
$     \Delta P =  
              \frac{  - \Delta{t} \overbrace{ \nabla \cdot \theta \vec{U}^{*^{f}} } ^{\text{Advection}(\theta, \vec{U}^{*^{f}}) } }
              {\kappa} $\\
where $P^{n+1} = P_{eq} + \Delta{P}$ for a single material $\theta = 1.0$
set boundary conditions on $P^{n+1}$
%
%__________________________________
\item \underline{Compute the face centered pressure}\\
$ P^{*^{f}} = \frac{\frac{P}{\rho} + \frac{P_{adj}}{\rho_{adj}}}
                     {\frac{1}{\rho} + \frac{1}{\rho_{adj}}}
              =\frac{ {P \rho_{adj}} + {P_{adj} \rho }  }
                 { {\rho} + {\rho_{adj}  }  } $
%
%__________________________________
\item \underline{Accumulate sources of momentum and internal energy}\\  
 $\Delta(m\vec{U}) = 
    - \delt V \nabla{ P^{*^f}} 
    + \nabla{ \cdot (\tau^{*^f}})
    + m \vec{g}\delt$
    
 $\Delta(me) =  
    V \kappa P \Delta{P}
    - \nabla q^{*^f}$

where $q^{*^f} = -k^f \nabla T$
%
%__________________________________
\item \underline{Compute Lagrangian quantities}\\
    $m^L     = {\rho}V$ \\
    $(m\U)^L = (m\U) + \Delta(m\U)$ \\
    $(m e)^L = (m e) + \Delta(m e)$ \\
%

%__________________________________
\item \underline{Advect and advance in time}\\
$ m^{n+1}     = m^L 
                  - \delt\text{Advection}(m^L, \U_ m^{*^{f}})$ \\
$(m \U)^{n+1} = (m \U)^L 
                   - \delt\text{Advection}((m \U)^L, \vec{U}^{*^{f}})$\\
$(me)^{n+1}   = (me)^L       
                   - \delt\text{Advection}((\rho  e)^L, \vec{U}^{*^{f}})$\\ %__________________________________
\item \underline{Compute primitive Variables from n+1 conserved quantities}\\
$ \rho^{n+1} =  m^{n+1}/V$\\
%
$\U^{n+1} = \frac{(m \U)^{n+1} }{ m^{n+1} }$\\
%
$T^{n+1} = \frac{(me)^{n+1} }{ m^{n+1} C_v }$ \\

set boundary conditions on the primitive variables $(\U, \rho, T)$.
\end{enumerate}
\end{document}
