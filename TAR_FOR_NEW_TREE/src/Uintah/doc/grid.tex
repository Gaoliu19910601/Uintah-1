
\section{3D AMR Grid Representation}

The responsibility of this library will be to manage the data in the
domain.  Data is represented on several {\em Patches}, collected into
{\em Levels} and formed into a {\em Hierarchy}.  Multiple {\em
Variables} exist over the entire computational domain.

\vspace{0.5in}

This is the basic interface to a hierarchy:
\begin{verbatim}
class Hierarchy {
public:
	Hierarchy(int maxLevels);
	~Hierarchy();
};
\end{verbatim}

In addition, we will need to define:
\begin{enumerate}
\item Iteration over levels
\item Operations on a composite grid
\item Computations that are performed on a composite grid
\end{enumerate}

\vspace{0.5in}


This is the basic interface to a Level:
\begin{verbatim}
class Level {
public:
	Level();
	~Level();


	Hierarchy* getHierarchy() const;	
	int levelNumber();
};
\end{verbatim}

In addition, we will need to define:
\begin{enumerate}
\item A mechanism to iterate over the patches in a level
\item A mechanism to perform parallel operations on a level
\item Computations that are performed at an entire level
\end{enumerate}


\vspace{0.5in}

This is the basic interface to a Patch:
\begin{verbatim}
class Patch {
public:
	Patch();
	~Patch();

	Level* getLevel() const;
	Point lowerCorner();
	Point upperCorner();
};

class StructuredPatch : public Patch {
};

class CurvilinearPatch : public Patch {
};

class UnstructuredPatch : public Patch {
};

\end{verbatim}

In addition, we will need to define:
\begin{enumerate}
\item A way to access data on the patch
\item A way to iterate over data (implicitly, explicitly, or both):
\begin{itemize}
\item Interior cells
\item All cells
\item Nodes
\item Faces
\item etc.
\end{itemize}
\item A way to map patches to procesors
\item A way to describe work to be done on patches
\item A way to relate patches to other patches
\end{enumerate}

\vspace{1.5in}

There will also be a basic ``Data'' class (or classes).  We will need
to define:
\begin{enumerate}
\item A way to express computation on these data classes
\item Perhaps a way to extract 3D pointers for passing to a legacy
      fortran code.
\item A way to express the ``type'' of data, including, face-centered
      vs. cell centered vs. vertex centered, and double, int, float,
      possibly units, etc.
\end{enumerate}

Specific computation requirements include:
\begin{enumerate}
\item Array operations (add/multiply, etc.)
\item Stencil operations
\item Array slicing/subsectioning
\item Tri-diagonal operations (arches) (ADI)
\item ``where''
\item do different stuff based on signs of velocities (for up-winding
      and for Todd's advection operator).
\item Norms
\item Generic linear and non-linear solver interfaces
\end{enumerate}

\vspace{1.5in}

Other responsibilities include:
\begin{enumerate}
\item Resource allocation
\item Managing of Hierarchy/regridding
\end{enumerate}
